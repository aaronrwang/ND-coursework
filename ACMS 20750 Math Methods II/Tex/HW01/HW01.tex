\documentclass{article}
\usepackage{amsmath, amsfonts, amssymb, amstext, mathtools,textcomp, xcolor}


\title{Homework 01}
\author{Aaron Wang}
\date{January 21 2025}

\begin{document}
\maketitle
\section{Monday 1/13}
\begin{enumerate}
%q1%
    \item Verify the statement of Example 2. Also verify that $y = \cosh(x)$ and $y = \sinh(x)x$ are solutions of $y''=y$.
    \begin{enumerate}
        \item $e^x$
        \textcolor{red}{
        \begin{center}
            $y=e^x$; $y'=e^x$; $y''=e^x$ so $y=y''$
        \end{center}
        }
        
        \item $e^{-x}$
        \textcolor{red}{
        \begin{center}
            $y=e^{-x}$; $y'=-e^{-x}$; $y''=e^{-x}$ so $y=y''$
        \end{center}
        }
        
        \item $Ae^x+Be^{-x}$
        \textcolor{red}{
        \begin{center}
            $y=Ae^x+Be^{-x}$; $y'=Ae^x-Be^{-x}$; $y''=Ae^x+Be^{-x}$ so $y=y''$
        \end{center}
        }
        
        \item $\cosh(x)$
        \textcolor{red}{
        \begin{center}
            $y=\cosh(x)$; $y'=\sinh(x)$; $y''=\cosh(x)$ so $y=y''$
        \end{center}
        }
        
        \item $\sinh(x)$
        \textcolor{red}{
        \begin{center}
            $y=\sinh(x)$; $y'=\cosh(x)$; $y''=\sinh(x)$ so $y=y''$
        \end{center}
        }
        
    \end{enumerate}
%q2%
    \item Find the solution of $y''=y$ which passes through the origin and through the point $(\ln2,\frac{3}{4})$.
    \vspace{5mm}\\
    The general solution of the differential equation is
    \[
    y = a \sinh x + b \cosh x
    \]
    \textcolor{red}{
    \[
        0 = a \sinh 0 + b \cosh 0 = a (0) + b(1) = b
    \]
    \[
        b = 0
    \]
    \[
        \frac{3}{4} = a \sinh (\ln2) + b \cosh (\ln2) = a (\frac{3}{4}) + b(\frac{5}{4}) = \frac{3}{4} a
    \]
    \[
        a = 1
    \]
    The desired particular solution is
    \[
        y = \sinh x
    \]
    }
\pagebreak
%q3%
    \item Verify that $y = \sin x$, $y = \cos x$, $y= e^{ix}$, and $y= e^{-ix}$ are all solutions of $y''=-y$.
    \begin{enumerate}
        \item $y = \sin x$
        \textcolor{red}{
        \begin{center}
            $y=\sin x$; $y'=\cos x$; $y''= - \sin x$ so $y''=-y$
        \end{center}
        }

        \item $y = \cos x$
        \textcolor{red}{
        \begin{center}
            $y=\cos x$; $y'= - \sin x$; $y''= - \cos x$ so $y''=-y$
        \end{center}
        }

        \item $e^{ix}$
        \textcolor{red}{
        \begin{center}
            $y=e^{ix}$; $y'=ie^{ix}$; $y''=i^2e^{-x}=-e^{ix}$ so $y=y''$
        \end{center}
        }

        \item $e^{-ix}$
        \textcolor{red}{
        \begin{center}
            $y=e^{-ix}$; $y'=-ie^{-ix}$; $y''=i^2e^{-x}=-e^{-ix}$ so $y=y''$
        \end{center}
        }
        
    \end{enumerate}
%q4%
    \item Find the distance which an object moves in time $t$ if it starts from rest and has an acceleration $\frac{d^2x}{dt^2} = ge^{-kt}$.
    \textcolor{red}{
    \[
    \frac{dx}{dt} = \int \frac{d^2x}{dt^2} dt= \int ge^{-kt} dt = -\frac{g}{k}e^{-kt}+C_1
    \]
    \[
    \frac{dx}{dt}\Big|_{x=0} = -\frac{g}{k}e^{-k(0)}+C_1 = 0 \text{ so } C_1 = \frac{g}{k}e^{-k(0)} = \frac{g}{k}
    \]
    \[
    x = \int \frac{dx}{dt} dt= \int -\frac{g}{k}e^{-kt}+C_1 dt = \frac{g}{k^2}e^{-kt}+C_1t+C_2
    \]
    \[
    x(0)= \frac{g}{k^2}e^{-k(0)}+C_1(0)+C_2 = 0 \text{ so } C_2 = -\Big(\frac{g}{k^2}e^{-k(0)}+C_1(0)\Big) = -\frac{g}{k^2}
    \]
    \[
    x(t) = \frac{g}{k^2}e^{-kt}+\frac{g}{k}t-\frac{g}{k^2}
    \]
    }
    Show that for small $t$ the result is approximately $(1.10)$($x=\frac{1}{2}gt^2$) 
    \textcolor{red}{ 
    % \[
    % x(t) = gk^2e^{-kt}+\frac{g}{k}t-\frac{g}{k^2} = \frac{g}{k^2}(e^{-kt}-1+kt)
    % \]
    % Observe:
    % \[
    % e^{-kt} = 1 + (-kt)+\frac{(-kt)^2}{2!}+\frac{(-kt)^3}{3!}+...
    % \]
    % Thus:
    % \[
    % x(t) = \frac{g}{k^2}\Bigg(\Big(1 + (-kt)+\frac{(-kt)^2}{2!}+\frac{(-kt)^3}{3!}+...\Big)-1+kt\Bigg) = \frac{g}{k^2}\Big(\frac{(-kt)^2}{2!}+\frac{(-kt)^3}{3!}+...\Big)
    % \]
    % For small $t$, higher order terms are negligible.
    % \[
    %     x(t) \approx\frac{g}{k^2}\Big(\frac{(-kt)^2}{2!}\Big) =\frac{1}{2}gt^2
    \[
        \lim_{t\rightarrow0} a = \lim_{t\rightarrow0} ge^{-kt} = g.
    \]
    Thus, when $t$ is small:
    \[
        v(t) \approx \int g = gt+C \text{ where } C = 0 \text{ because }v(0) = 0
    \]
    \[
        x(t) \approx \int gt = \frac{1}{2}gt^2+C \text{ where } C = 0 \text{ because }x(0) = 0
    \]
    Thus, 
    \[
        x(t) \approx  \frac{1}{2}gt^2 \text { for small t}
    \]
    % \]
    }
    Show for very large $t$, the speed $\frac{dx}{dt}$ is approximately constant.
    \textcolor{red}{
    \[
    \lim_{t \rightarrow \infty}\frac{dx}{dt} = \lim_{t \rightarrow \infty} \Big(-\frac{g}{k}e^{-kt}+\frac{g}{k}\Big) = \frac{g}{k}
    \]
    }
\pagebreak
%q5%
    \item Find the position $x$ of a particle at time $t$ if its acceleration is $\frac{d^2x}{dt^2} = A \sin (\omega t)$.
    \textcolor{red}{
    \[
    \frac{dx}{dt}=\int\frac{d^2x}{dt^2}dt = \int A \sin (\omega t)dt=-A\omega^{-1} \cos (\omega t) + C_1
    \]
    \[
    x=\int\frac{dx}{dt}dt = -\int A\omega \cos (\omega t) + C_1 dt = -A\omega^{-2} \sin (\omega t) + C_1t+C_2
    \]
    }
\end{enumerate}
\section{Wednesday 1/15}
For each of the following differential equations, separate variables and find a solution
containing one arbitrary constant. Then find the value of the constant to give a particular solution satisfying the given boundary condition.
\begin{enumerate}
    \item[1.] $z^3$
\begin{align*}
    u(x,y) &= x^3 - 3xy^2\\
    v(x,y) &= 3xy^2 - y^3
\end{align*}
\begin{align*}
    \pd{u}{x}  &= 3x^2 - 3y^2  &\qquad \pd{v}{y}  &= 3x^2 - 3y^2 \\
    \pd{u}{y} &= -6xy           &\qquad \pd{v}{x} &= 6xy
\end{align*}

\answer{
Since $\pd{u}{x} = \pd{v}{y}$ and $-\pd{u}{y} = \pd{v}{x}$, $f(z) = z^3$ is analytical.
}

    \item $x\sqrt{1-y^2}dx+y\sqrt{1-x^2}dy=0$, $y = \frac{1}{2}$ when $x = \frac{1}{2}$
    
    \textcolor{blue}{
    \begin{minipage}[t]{0.45\textwidth}
        General Solution:
        \begin{gather*}
            x\sqrt{1-y^2}dx+y\sqrt{1-x^2}dy = 0 \\
            y\sqrt{1-x^2}dy = -x\sqrt{1-y^2}dx \\
            \frac{y}{\sqrt{1-y^2}}dy = -\frac{x}{\sqrt{1-x^2}}dx \\
            \int\frac{y}{\sqrt{1-y^2}}dy = -\int\frac{x}{\sqrt{1-x^2}}dx \\
            -\frac{1}{2}\sqrt{1-y^2} = \frac{1}{2}\sqrt{1-x^2}+C \\
            \frac{1}{2}\sqrt{1-y^2} + \frac{1}{2}\sqrt{1-x^2} = C \\
            \textcolor{red}{
            \sqrt{1-y^2} + \sqrt{1-x^2} = C   
            }
        \end{gather*}
    \end{minipage}
    \hfill
    \begin{minipage}[t]{0.45\textwidth}
        Integration:\\
            $u = 1-t^2$; $du = -2tdt$; $dt = -\frac{1}{2t}du$      
        \[
            \int\frac{t}{\sqrt{1-t^2}}dt =\int\frac{t}{\sqrt{u}}\frac{-1}{2t}du
        \]
        \[
            =-\frac{1}{2}\int\frac{1}{\sqrt{u}}du
            = -\frac{1}{2}(\sqrt{u})+C
        \]
        \[= -\frac{1}{2}\sqrt{1-t^2}+C\]
        Particular Solution:
        \begin{gather*}
            \sqrt{1-(1/2)^2} + \sqrt{1-(1/2)^2} = C \\
            C = \sqrt{3/4}+\sqrt{3/4} = \sqrt3 \\
            \textcolor{red}{
            \sqrt{1-y^2} + \sqrt{1-x^2} = \sqrt3
            }
        \end{gather*}
    \end{minipage}
    }
    \item[3.] $\overline{z}$
\begin{align*}
    u(x,y) &= x\\
    v(x,y) &= -y
\end{align*}
\begin{align*}
    \pd{u}{x}  &= 1  &\qquad \pd{v}{y}  &= -1
\end{align*}
\answer{
Since $\pd{u}{x} \neq \pd{v}{y}$, $f(z) = \overline{z}$ is NOT analytical.
}

    \item[4.] $|z|$
\begin{align*}
    u(x,y) &= \sqrt{x^2+y^2}\\
    v(x,y) &= 0
\end{align*}
\begin{align*}
    \pd{u}{x}  &= 2x\sqrt{x^2+y^2}  &\qquad \pd{v}{y}  &= 0
\end{align*}
\answer{
Since $\pd{u}{x} \neq \pd{v}{y}$, $f(z) = |z|$ is NOT analytical.
}

\pagebreak
    \item[5.] Re$(z)$
\begin{align*}
    u(x,y) &= x\\
    v(x,y) &= 0
\end{align*}
\begin{align*}
    \pd{u}{x}  &= 1  &\qquad \pd{v}{y}  &= 0
\end{align*}
\answer{
Since $\pd{u}{x} \neq \pd{v}{y}$, $f(z) = \text{Re}(z)$ is NOT analytical.
}

    \item $y'=\frac{2xy^2 + x}{x^2y-y}$, $y=0$ when $x=\sqrt2$

\textcolor{blue}{
    \begin{minipage}[t]{0.45\textwidth}
        General Solution:
        \begin{gather*}
            \frac{dy}{dx}=\frac{x(2y^2 + 1)}{y(x^2-1)}\\
            \frac{y}{2y^2 + 1}dy=\frac{x}{x^2-1}dx\\
            \int\frac{y}{2y^2 + 1}dy=\int\frac{x}{x^2-1}dx\\
            \frac{1}{4}\ln|2y^2 + 1|=\frac{1}{2}\ln|x^2 - 1|+C\\
            \ln|2y^2 + 1|=2\ln|x^2 - 1|+C\\
            e^{\ln|2y^2 + 1|}=e^{2\ln|x^2 - 1|+C}\\
            \textcolor{red}{
                2y^2 + 1=C(x^2 - 1)^2
            }
        \end{gather*}
        Particular Solution:
        \begin{gather*}
            2(0)^2 + 1=C((\sqrt2)^2 - 1)^2\\
            1=C(2 - 1)^2\\
            C = 1\\
            \textcolor{red}{
                2y^2 + 1=(x^2 - 1)^2
            }
        \end{gather*}
    \end{minipage}
    \hfill
    \begin{minipage}[t]{0.45\textwidth}
        Integration for LHS:\\
        $u=2y^2 + 1$; $du = 4ydy$
        \begin{gather*}
            \int\frac{y}{2y^2 + 1}dy = \frac{1}{4}\int\frac{1}{u}du\\
            = \frac{1}{4}\ln|u|+C=\frac{1}{4}\ln|2y^2 + 1|+C
        \end{gather*}
        Integration for RHS:\\
        $u=x^2 - 1$; $du = 2xdx$
        \begin{gather*}
            \int\frac{x}{x^2-1}dx = \frac{1}{2}\int\frac{1}{u}du\\
            = \frac{1}{2}\ln|u|+C=\frac{1}{2}\ln|x^2 - 1|+C
        \end{gather*}
    \end{minipage}
}
\pagebreak
    \item[7.] $\cosh(z)$
\begin{align*}
    u(x,y) &= \cosh(x)\cos(y)\\
    v(x,y) &= \sinh(x)\sin(y)
\end{align*}
\begin{align*}
    \pd{u}{x}  &= \sinh(x)\cos(y)  &\qquad \pd{v}{y}  &= \sinh(x)\cos(y) \\
    \pd{u}{y}  &= -\cosh(x)\sin(y)  &\qquad \pd{v}{x}  &= \cosh(x)\sin(y) 
\end{align*}
\answer{
Since $\pd{u}{x} = \pd{v}{y}$ and $-\pd{u}{y} = \pd{v}{x}$, $f(z) = \cosh(z)$ is analytical.
}

    \item $y' + 2xy^2 = 0$, $y=1$ when $x=2$

\textcolor{blue}{
    \begin{minipage}[t]{0.45\textwidth}
        General Solution:
        \begin{gather*}
            \frac{dy}{dx} + 2xy^2 = 0\\
            \frac{dy}{dx} =- 2xy^2 \\
            -\frac{dy}{y^2} = 2xdx \\
            \int-\frac{dy}{y^2} = \int2xdx \\
            \frac{1}{y}=x^2+C\\
            \textcolor{red}{
            y=\frac{1}{x^2+C}
            }
        \end{gather*}
    \end{minipage}
    \hfill
    \begin{minipage}[t]{0.45\textwidth}
        Particular Solution:
        \begin{gather*}
            1=\frac{1}{(2)^2+C}\\
            4+C=1\\
            C=-1\\
            \textcolor{red}{
                y=\frac{1}{x^2-3}
            }
        \end{gather*}
    \end{minipage}
}
\pagebreak
    \item[9.] $\frac{1}{z}$
\begin{align*}
    u(x,y) &= \frac{x}{x^2+y^2}\\
    v(x,y) &= \frac{-y}{x^2+y^2}
\end{align*}
\begin{align*}
    \pd{u}{x}  &= \frac{-x^2+y^2}{(x^2+y^2)^2}  
    &\qquad 
    \pd{v}{y}  &= \frac{-x^2+y^2}{(x^2+y^2)^2} \\
    \pd{u}{y}  &= \frac{-2xy}{(x^2+y^2)^2}  
    &\qquad 
    \pd{v}{x}  &= \frac{2xy}{(x^2+y^2)^2} 
\end{align*}
\answer{
Since $\pd{u}{x} = \pd{v}{y}$ and $-\pd{u}{y} = \pd{v}{x}$, $f(z) = \frac{1}{z}$ is analytical (assuming $z \neq 0$).
}

    \item $y'-xy=x$, $y=1$ when $x=0$

\textcolor{blue}{
    \begin{minipage}[t]{0.45\textwidth}
        General Solution:
        \begin{gather*}
        \frac{dy}{dx}-xy=x\\
        \frac{dy}{dx}=xy+x\\
        \frac{dy}{dx}=x(y+1)\\
        \frac{dy}{y+1}=xdx\\
        \int\frac{dy}{y+1}=\int xdx\\
        \ln|y+1|=\frac{x^2}{2}+C\\
        e^{\ln|y+1|}=e^{x^2/2+C}\\
        y+1=Ce^{x^2/2}\\
        \textcolor{red}{
            y=Ce^{x^2/2}-1
        }
        \end{gather*}
    \end{minipage}
    \hfill
    \begin{minipage}[t]{0.45\textwidth}
        Particular Solution:
        \begin{gather*}
        (1)=Ce^{(0)^2/2}-1\\
        C=2\\
        \textcolor{red}{
            y=2e^{x^2/2}-1
        }
        \end{gather*}
    \end{minipage}
}
\pagebreak
    \item[11.] $\frac{2z-i}{iz+2}$
\begin{align*}
    u(x,y) &= \frac{3x}{x^2+(y-2)^2} \\
    v(x,y) &= \frac{-2x^2-2y^2+5y-2}{x^2+(y-2)^2}
\end{align*}
\begin{align*}
    \pd{u}{x}  
    &= \frac{(3)(x^2+(y-2)^2)-(3x)(2x)}{(x^2+(y-2)^2)^2} \\
    &= \frac{3x^2+3(y-2)^2-6x^2}{(x^2+(y-2)^2)^2} \\
    &= \frac{-3x^2+3y^2-12y+12}{(x^2+(y-2)^2)^2} \\
    \pd{v}{y}  
    &= \frac{(-4y+5)(x^2+(y-2)^2)-(-2x^2-2y^2+5y-2)(2(y-2))}{(x^2+(y-2)^2)^2} \\
    &= \frac{(-4y+5)(x^2+(y-2)^2)-(-4x^2y-4y^3+10y^2-4y)+(-8x^2-8y^2+20y-8)}{(x^2+(y-2)^2)^2} \\
    &= \frac{(-4y+5)(x^2+(y-2)^2)+4x^2y+4y^3-10y^2+4y-8x^2-8y^2+20y-8}{(x^2+(y-2)^2)^2} \\
    &= \frac{(-4y+5)(x^2+(y-2)^2)+4x^2y+4y^3-18y^2+24y-8x^2-8}{(x^2+(y-2)^2)^2} \\
    &= \frac{-4x^2y+-4y(y-2)^2+5x^2+5(y-2)^2+4x^2y+4y^3-18y^2+24y-8x^2-8}{(x^2+(y-2)^2)^2} \\
    &= \frac{-4y(y-2)^2+5(y-2)^2+4y^3-18y^2+24y-3x^2-8}{(x^2+(y-2)^2)^2} \\
    &= \frac{-4y(y-2)^2+4y^3-13y^2+4y-3x^2+12}{(x^2+(y-2)^2)^2} \\
    &= \frac{-3x^2+3y^2-12y+12}{(x^2+(y-2)^2)^2} \\
    \pd{u}{y}  
    &= \frac{(0)(x^2+(y-2)^2)-(3x)(2(y-2))}{(x^2+(y-2)^2)^2} \\
    &= \frac{-6xy+12x}{(x^2+(y-2)^2)^2} \\
\end{align*}
\begin{align*}
    \pd{v}{x}  
    &= \frac{(-4x)(x^2+(y-2)^2-(-2x^2-2y^2+5y-2)(2x)}{(x^2+y^2)^2} \\
    &= \frac{-4x^3-4xy^2+16xy-16x+4x^3+4xy^2-10xy+4x}{(x^2+y^2)^2} \\
    &= \frac{16xy-16x-10xy+4x}{(x^2+y^2)^2} \\
    &= \frac{6xy-12x}{(x^2+y^2)^2} \\
\end{align*}
\answer{
Since $\pd{u}{x} = \pd{v}{y}$ and $-\pd{u}{y} = \pd{v}{x}$, $f(z) = \frac{2z-i}{iz+2}$ is analytical (assuming $z \neq 2i$).
}

    \item $(x + xy)y' + y = 0$

\textcolor{blue}{
    \begin{minipage}[t]{0.45\textwidth}
        General Solution:
        \begin{gather*}
        (x + xy)\frac{dy}{dx}= -y\\
        x(1 + y)\frac{dy}{dx}= -y\\
        \frac{1 + y}{y}dy= -xdx\\
        \Big(\frac{1}{y}+1\Big)dy= -\frac{1}{x}dx\\
        \int \Big(\frac{1}{y}+1\Big)dy= -\int \frac{1}{x}dx\\
        \ln|y| + y = -\ln|x|+C \\
        e^{\ln|y| + y} = e^{-\ln|x|+C} \\
        \textcolor{red}{
        ye^y=\frac{C}{x}
        }
        \end{gather*}
    \end{minipage}
    \hfill
    \begin{minipage}[t]{0.45\textwidth}
        Particular Solution:
        \begin{gather*}
        (1)e^{(1)}=\frac{C}{(1)}\\
        C=e\\
        \textcolor{red}{
        ye^y=\frac{e}{x}
        }
        \end{gather*}
    \end{minipage}
}    
\end{enumerate}

\end{document}