\documentclass{article}
\usepackage{amsmath, amsfonts, amssymb, amstext, mathtools,textcomp, xcolor}


\title{Homework 01}
\author{Aaron Wang}
\date{January 21 2025}

\begin{document}
\maketitle
\section{Monday 1/13}
\begin{enumerate}
%q1%
    \item Verify the statement of Example 2. Also verify that $y = \cosh(x)$ and $y = \sinh(x)x$ are solutions of $y''=y$.
    \begin{enumerate}
        \item $e^x$
        \textcolor{red}{
        \begin{center}
            $y=e^x$; $y'=e^x$; $y''=e^x$ so $y=y''$
        \end{center}
        }
        
        \item $e^{-x}$
        \textcolor{red}{
        \begin{center}
            $y=e^{-x}$; $y'=-e^{-x}$; $y''=e^{-x}$ so $y=y''$
        \end{center}
        }
        
        \item $Ae^x+Be^{-x}$
        \textcolor{red}{
        \begin{center}
            $y=Ae^x+Be^{-x}$; $y'=Ae^x-Be^{-x}$; $y''=Ae^x+Be^{-x}$ so $y=y''$
        \end{center}
        }
        
        \item $\cosh(x)$
        \textcolor{red}{
        \begin{center}
            $y=\cosh(x)$; $y'=\sinh(x)$; $y''=\cosh(x)$ so $y=y''$
        \end{center}
        }
        
        \item $\sinh(x)$
        \textcolor{red}{
        \begin{center}
            $y=\sinh(x)$; $y'=\cosh(x)$; $y''=\sinh(x)$ so $y=y''$
        \end{center}
        }
        
    \end{enumerate}
%q2%
    \item Find the solution of $y''=y$ which passes through the origin and through the point $(\ln2,\frac{3}{4})$.
    \vspace{5mm}\\
    The general solution of the differential equation is
    \[
    y = a \sinh x + b \cosh x
    \]
    \textcolor{red}{
    \[
        0 = a \sinh 0 + b \cosh 0 = a (0) + b(1) = b
    \]
    \[
        b = 0
    \]
    \[
        \frac{3}{4} = a \sinh (\ln2) + b \cosh (\ln2) = a (\frac{3}{4}) + b(\frac{5}{4}) = \frac{3}{4} a
    \]
    \[
        a = 1
    \]
    The desired particular solution is
    \[
        y = \sinh x
    \]
    }
\pagebreak
%q3%
    \item Verify that $y = \sin x$, $y = \cos x$, $y= e^{ix}$, and $y= e^{-ix}$ are all solutions of $y''=-y$.
    \begin{enumerate}
        \item $y = \sin x$
        \textcolor{red}{
        \begin{center}
            $y=\sin x$; $y'=\cos x$; $y''= - \sin x$ so $y''=-y$
        \end{center}
        }

        \item $y = \cos x$
        \textcolor{red}{
        \begin{center}
            $y=\cos x$; $y'= - \sin x$; $y''= - \cos x$ so $y''=-y$
        \end{center}
        }

        \item $e^{ix}$
        \textcolor{red}{
        \begin{center}
            $y=e^{ix}$; $y'=ie^{ix}$; $y''=i^2e^{-x}=-e^{ix}$ so $y=y''$
        \end{center}
        }

        \item $e^{-ix}$
        \textcolor{red}{
        \begin{center}
            $y=e^{-ix}$; $y'=-ie^{-ix}$; $y''=i^2e^{-x}=-e^{-ix}$ so $y=y''$
        \end{center}
        }
        
    \end{enumerate}
%q4%
    \item Find the distance which an object moves in time $t$ if it starts from rest and has an acceleration $\frac{d^2x}{dt^2} = ge^{-kt}$.
    \textcolor{red}{
    \[
    \frac{dx}{dt} = \int \frac{d^2x}{dt^2} dt= \int ge^{-kt} dt = -\frac{g}{k}e^{-kt}+C_1
    \]
    \[
    \frac{dx}{dt}\Big|_{x=0} = -\frac{g}{k}e^{-k(0)}+C_1 = 0 \text{ so } C_1 = \frac{g}{k}e^{-k(0)} = \frac{g}{k}
    \]
    \[
    x = \int \frac{dx}{dt} dt= \int -\frac{g}{k}e^{-kt}+C_1 dt = \frac{g}{k^2}e^{-kt}+C_1t+C_2
    \]
    \[
    x(0)= \frac{g}{k^2}e^{-k(0)}+C_1(0)+C_2 = 0 \text{ so } C_2 = -\Big(\frac{g}{k^2}e^{-k(0)}+C_1(0)\Big) = -\frac{g}{k^2}
    \]
    \[
    x(t) = \frac{g}{k^2}e^{-kt}+\frac{g}{k}t-\frac{g}{k^2}
    \]
    }
    Show that for small $t$ the result is approximately $(1.10)$($x=\frac{1}{2}gt^2$) 
    \textcolor{red}{ 
    % \[
    % x(t) = gk^2e^{-kt}+\frac{g}{k}t-\frac{g}{k^2} = \frac{g}{k^2}(e^{-kt}-1+kt)
    % \]
    % Observe:
    % \[
    % e^{-kt} = 1 + (-kt)+\frac{(-kt)^2}{2!}+\frac{(-kt)^3}{3!}+...
    % \]
    % Thus:
    % \[
    % x(t) = \frac{g}{k^2}\Bigg(\Big(1 + (-kt)+\frac{(-kt)^2}{2!}+\frac{(-kt)^3}{3!}+...\Big)-1+kt\Bigg) = \frac{g}{k^2}\Big(\frac{(-kt)^2}{2!}+\frac{(-kt)^3}{3!}+...\Big)
    % \]
    % For small $t$, higher order terms are negligible.
    % \[
    %     x(t) \approx\frac{g}{k^2}\Big(\frac{(-kt)^2}{2!}\Big) =\frac{1}{2}gt^2
    \[
        \lim_{t\rightarrow0} a = \lim_{t\rightarrow0} ge^{-kt} = g.
    \]
    Thus, when $t$ is small:
    \[
        v(t) \approx \int g = gt+C \text{ where } C = 0 \text{ because }v(0) = 0
    \]
    \[
        x(t) \approx \int gt = \frac{1}{2}gt^2+C \text{ where } C = 0 \text{ because }x(0) = 0
    \]
    Thus, 
    \[
        x(t) \approx  \frac{1}{2}gt^2 \text { for small t}
    \]
    % \]
    }
    Show for very large $t$, the speed $\frac{dx}{dt}$ is approximately constant.
    \textcolor{red}{
    \[
    \lim_{t \rightarrow \infty}\frac{dx}{dt} = \lim_{t \rightarrow \infty} \Big(-\frac{g}{k}e^{-kt}+\frac{g}{k}\Big) = \frac{g}{k}
    \]
    }
\pagebreak
%q5%
    \item Find the position $x$ of a particle at time $t$ if its acceleration is $\frac{d^2x}{dt^2} = A \sin (\omega t)$.
    \textcolor{red}{
    \[
    \frac{dx}{dt}=\int\frac{d^2x}{dt^2}dt = \int A \sin (\omega t)dt=-A\omega^{-1} \cos (\omega t) + C_1
    \]
    \[
    x=\int\frac{dx}{dt}dt = -\int A\omega \cos (\omega t) + C_1 dt = -A\omega^{-2} \sin (\omega t) + C_1t+C_2
    \]
    }
\end{enumerate}
\section{Wednesday 1/15}
For each of the following differential equations, separate variables and find a solution
containing one arbitrary constant. Then find the value of the constant to give a particular solution satisfying the given boundary condition.
\begin{enumerate}
    \begin{figure}[h]
\centering
\begin{tikzpicture}
    \node[state, initial] (q1) {1};
    \node[state, right of=q1, xshift=1cm] (q2) {2};
    \node[state, right of=q2, xshift=1cm] (q3) {3};
    \node[state, accepting, right of=q3, xshift=1cm] (q4) {4};
    \draw
    (q1) edge[loop above] node[above]{\str{0}, \str{1}} (q1)
    (q1) edge node[above]{\str{1}} (q2)
    (q2) edge node[above]{\str{0}, \str{1}} (q3)
    (q3) edge node[above]{\str{0}, \str{1}} (q4)
    ;
\end{tikzpicture}
\end{figure}
    \item $x\sqrt{1-y^2}dx+y\sqrt{1-x^2}dy=0$, $y = \frac{1}{2}$ when $x = \frac{1}{2}$
    
    \textcolor{blue}{
    \begin{minipage}[t]{0.45\textwidth}
        General Solution:
        \begin{gather*}
            x\sqrt{1-y^2}dx+y\sqrt{1-x^2}dy = 0 \\
            y\sqrt{1-x^2}dy = -x\sqrt{1-y^2}dx \\
            \frac{y}{\sqrt{1-y^2}}dy = -\frac{x}{\sqrt{1-x^2}}dx \\
            \int\frac{y}{\sqrt{1-y^2}}dy = -\int\frac{x}{\sqrt{1-x^2}}dx \\
            -\frac{1}{2}\sqrt{1-y^2} = \frac{1}{2}\sqrt{1-x^2}+C \\
            \frac{1}{2}\sqrt{1-y^2} + \frac{1}{2}\sqrt{1-x^2} = C \\
            \textcolor{red}{
            \sqrt{1-y^2} + \sqrt{1-x^2} = C   
            }
        \end{gather*}
    \end{minipage}
    \hfill
    \begin{minipage}[t]{0.45\textwidth}
        Integration:\\
            $u = 1-t^2$; $du = -2tdt$; $dt = -\frac{1}{2t}du$      
        \[
            \int\frac{t}{\sqrt{1-t^2}}dt =\int\frac{t}{\sqrt{u}}\frac{-1}{2t}du
        \]
        \[
            =-\frac{1}{2}\int\frac{1}{\sqrt{u}}du
            = -\frac{1}{2}(\sqrt{u})+C
        \]
        \[= -\frac{1}{2}\sqrt{1-t^2}+C\]
        Particular Solution:
        \begin{gather*}
            \sqrt{1-(1/2)^2} + \sqrt{1-(1/2)^2} = C \\
            C = \sqrt{3/4}+\sqrt{3/4} = \sqrt3 \\
            \textcolor{red}{
            \sqrt{1-y^2} + \sqrt{1-x^2} = \sqrt3
            }
        \end{gather*}
    \end{minipage}
    }
    \item[3.] $\overline{z}$
\begin{align*}
    u(x,y) &= x\\
    v(x,y) &= -y
\end{align*}
\begin{align*}
    \pd{u}{x}  &= 1  &\qquad \pd{v}{y}  &= -1
\end{align*}
\answer{
Since $\pd{u}{x} \neq \pd{v}{y}$, $f(z) = \overline{z}$ is NOT analytical.
}

    \item $(1 + y^2) dx + xy dy = 0$, $y = 0$ when $x = 5$.

\textcolor{blue}{
    \begin{minipage}[t]{0.45\textwidth}
        General Solution:
        \begin{gather*}
            (1 + y^2) dx + xy dy = 0 \\
            (1 + y^2) dx = - xy dy \\
            \frac{y}{1+y^2}dy = -\frac{1}{x}dx\\
            \int\frac{y}{1+y^2}dy = -\int\frac{1}{x}dx\\
            \frac{1}{2}\ln|1+y^2| = -\ln|x|+C \\
            e^{\frac{1}{2}\ln|1+y^2|} = e^{-\ln|x|+C}\\
            \sqrt{1+y^2}=\frac{C}{x}\\
            \textcolor{red}{
            1+y^2=\frac{C}{x^2}
            }
        \end{gather*}
    \end{minipage}
    \hfill
    \begin{minipage}[t]{0.45\textwidth}
        Integration for LHS:\\
        $u=1+y^2$; $du=2ydy$
        \begin{gather*}
            \int\frac{y}{1+y^2}dy = \frac{1}{2}\int\frac{1}{u}du \\
            =\frac{1}{2}\ln|u|+C=\frac{1}{2}\ln|1+y^2|+C
        \end{gather*}
        Particular Solution:
        \begin{gather*}
            1+(0)^2=\frac{C}{(5)^2}\\
            C=25\\
            \textcolor{red}{
            1+y^2=\frac{25}{x^2}
            }
        \end{gather*}
    \end{minipage}
}
\pagebreak
    \item $xy' - xy = y$, $y = 1$ when $x = 1$

\textcolor{blue}{
    \begin{minipage}[t]{0.45\textwidth}
        General Solution:
        \begin{gather*}
        x\frac{dy}{dx} - xy = y\\
        x\frac{dy}{dx}  = y+xy\\
        x\frac{dy}{dx}  = y(1+x)\\
        \frac{dy}{y}  = (\frac{1}{x}+1)dx\\
        \int\frac{dy}{y}  = \int(\frac{1}{x}+1)dx\\
        \ln|y| = \ln|x|+x+C \\
        e^{\ln|y|} = e^{\ln|x|+x+C}\\
        \textcolor{red}{
        y=Cxe^x
        }
        \end{gather*}
    \end{minipage}
    \hfill
    \begin{minipage}[t]{0.45\textwidth}
        Particular Solution:
        \begin{gather*}
            (1)=C(1)e^{(1)}\\
            C=\frac{1}{e}\\
            \textcolor{red}{
            y=\frac{1}{e}xe^x
            }
        \end{gather*}
    \end{minipage}
}
    \item[6.] $e^z$
\begin{align*}
    u(x,y) &= e^x\cos(y)\\
    v(x,y) &= e^x\sin(y)
\end{align*}
\begin{align*}
    \pd{u}{x}  &= e^x\cos(y)  &\qquad \pd{v}{y}  &= e^x\cos(y)\\ 
    \pd{u}{y}  &= -e^x\sin(y)  &\qquad \pd{v}{x}  &= e^x\sin(y) 
\end{align*}
\answer{
Since $\pd{u}{x} = \pd{v}{y}$ and $-\pd{u}{y} = \pd{v}{x}$, $f(z) = e^z$ is analytical.
}

\pagebreak
    \item $y dy + (xy^2 - 8x) dx = 0$, $y=3$ when $x=1$

\textcolor{blue}{
    \begin{minipage}[t]{0.45\textwidth}
        General Solution:
        \begin{gather*}
            y dy + (xy^2 - 8x) dx = 0\\
            -y dy = (xy^2 - 8x) dx\\
            -y dy = x(y^2-8) dx\\
            -\frac{y}{y^2-8} dy = x dx\\
            -\int\frac{y}{y^2-8} dy = \int x dx\\
            -\frac{1}{2}\ln|y^2-8|=\frac{x^2}{2}+C\\
            \ln|y^2-8|=-x^2+C\\
            e^{\ln|y^2-8|}=e^{-x^2+C}\\
            \textcolor{red}{
            y^2-8=Ce^{-x^2}
            }
        \end{gather*}
    \end{minipage}
    \hfill
    \begin{minipage}[t]{0.45\textwidth}
        Integration for LHS:\\
        $u=y^2-8$; $du=2ydy$
        \begin{gather*}
            \int\frac{y}{y^2-8} dy = \frac{1}{2}\int\frac{1}{u} dy\\
            \frac{1}{2}\ln|u|+C=\frac{1}{2}\ln|y^2-8|+C
        \end{gather*}
        Particular Solution:
        \begin{gather*}
            (3)^2-8=Ce^{-(1)^2}\\
            9-8=C/e\\
            C=e\\
            \textcolor{red}{
                y^2-8=e \cdot e^{-x^2}
            }
        \end{gather*}
    \end{minipage}
}
    \item $y' + 2xy^2 = 0$, $y=1$ when $x=2$

\textcolor{blue}{
    \begin{minipage}[t]{0.45\textwidth}
        General Solution:
        \begin{gather*}
            \frac{dy}{dx} + 2xy^2 = 0\\
            \frac{dy}{dx} =- 2xy^2 \\
            -\frac{dy}{y^2} = 2xdx \\
            \int-\frac{dy}{y^2} = \int2xdx \\
            \frac{1}{y}=x^2+C\\
            \textcolor{red}{
            y=\frac{1}{x^2+C}
            }
        \end{gather*}
    \end{minipage}
    \hfill
    \begin{minipage}[t]{0.45\textwidth}
        Particular Solution:
        \begin{gather*}
            1=\frac{1}{(2)^2+C}\\
            4+C=1\\
            C=-1\\
            \textcolor{red}{
                y=\frac{1}{x^2-3}
            }
        \end{gather*}
    \end{minipage}
}
\pagebreak
    \item $(1 + y)y'=y$, $y=1$ when $x=1$

\textcolor{blue}{
    \begin{minipage}[t]{0.45\textwidth}
        General Solution:
        \begin{gather*}
        (1 + y)\frac{dy}{dx}=y\\
        \Big(\frac{1}{y}+1\Big)dy=dx\\
        \int\Big(\frac{1}{y}+1\Big)dy=\int dx\\
        \ln|y|+y=x+C\\
        e^{\ln|y|+y}=e^{x+C}\\
        \textcolor{red}{
        ye^y=Ce^x
        }
        \end{gather*}
    \end{minipage}
    \hfill
    \begin{minipage}[t]{0.45\textwidth}
        Particular Solution:
        \begin{gather*}
        (1)e^(1)=Ce^(1)\\
        C=1\\
        \textcolor{red}{
        ye^y=e^x
        }
        \end{gather*}
    \end{minipage}
}
    \item $y'-xy=x$, $y=1$ when $x=0$

\textcolor{blue}{
    \begin{minipage}[t]{0.45\textwidth}
        General Solution:
        \begin{gather*}
        \frac{dy}{dx}-xy=x\\
        \frac{dy}{dx}=xy+x\\
        \frac{dy}{dx}=x(y+1)\\
        \frac{dy}{y+1}=xdx\\
        \int\frac{dy}{y+1}=\int xdx\\
        \ln|y+1|=\frac{x^2}{2}+C\\
        e^{\ln|y+1|}=e^{x^2/2+C}\\
        y+1=Ce^{x^2/2}\\
        \textcolor{red}{
            y=Ce^{x^2/2}-1
        }
        \end{gather*}
    \end{minipage}
    \hfill
    \begin{minipage}[t]{0.45\textwidth}
        Particular Solution:
        \begin{gather*}
        (1)=Ce^{(0)^2/2}-1\\
        C=2\\
        \textcolor{red}{
            y=2e^{x^2/2}-1
        }
        \end{gather*}
    \end{minipage}
}
\pagebreak
    \item $2y' = 3(y - 2)^{1/3}$, $y=3$ when $x=1$

\textcolor{blue}{
    \begin{minipage}[t]{0.45\textwidth}
        General Solution:
        \begin{gather*}
            2\frac{dy}{dx} = 3(y - 2)^{1/3}\\
            \frac{2}{3}\frac{1}{(y - 2)^{1/3}}dy =dx\\
            \frac{2}{3}\int\frac{1}{(y - 2)^{1/3}}dy=\int dx\\
            \frac{2}{3}\frac{3}{2}(y-2)^{2/3}=x+C\\
            \textcolor{red}{
            (y-2)^{2/3}=x+C
            }
        \end{gather*}
    \end{minipage}
    \hfill
    \begin{minipage}[t]{0.45\textwidth}
        Integration for LHS:\\
        $u=y-2$; $du=dy$
        \begin{gather*}
            \int\frac{1}{(y - 2)^{1/3}}dy=\int(u)^{-1/3}du\\
            \frac{3}{2}u^{2/3}+C=\frac{3}{2}(y-2)^{2/3}+C
        \end{gather*}
        Particular Solution:
        \begin{gather*}
            ((3)-2)^{2/3}=(1)+C\\
            1=1+C\\
            C=0\\
            \textcolor{red}{
                (y-2)^{2/3}=x
            }
        \end{gather*}
    \end{minipage}
}
    \item[12.] $\frac{z}{z^2+1}$
\begin{align*}
    u(x,y) &= \frac{x^3 + x + x y^2}{(x^2 - y^2 + 1)^2 + 4x^2 y^2} \\
    v(x,y) &= \frac{-x^2 y - y^3 + y}{(x^2 - y^2 + 1)^2 + 4x^2 y^2}
\end{align*}
\begin{align*}
    \pd{u}{x} &= \frac{-x^6-x^4y^2-x^4+x^2y^4-10x^2y^2+x^2+y^6-y^4-y^2+1}{\left(x^4+2x^2y^2+2x^2+y^4-2y^2+1\right)^2} \\
    \pd{v}{y} &= \frac{y^6+y^4x^2-y^4-y^2x^4-10y^2x^2-y^2-x^6-x^4+x^2+1}{\left(y^4+2y^2x^2-2y^2+x^4+2x^2+1\right)^2} \\
    \pd{u}{y} &= \frac{2y^5x+4y^3x^3+4y^3x+2yx^5-4yx^3-6yx}{\left(y^4+2y^2x^2-2y^2+x^4+2x^2+1\right)^2} \\
    \pd{v}{x} &= \frac{-2x^5y-4x^3y^3+4x^3y-2xy^5-4xy^3+6xy}{\left(x^4+2x^2y^2+2x^2+y^4-2y^2+1\right)^2}
\end{align*}
\answer{
Since $\pd{u}{x} = \pd{v}{y}$ and $-\pd{u}{y} = \pd{v}{x}$, $f(z) = \frac{2z-i}{iz+2}$ is analytical (assuming $z \neq \pm i$).
}


    
\end{enumerate}

\end{document}