\item [27.] According to Newton’s law of cooling, the rate at which the temperature of an object changes is proportional to the difference between its temperature and that of its surroundings. A cup of coffee at $200 \degree$ in a room of temperature $70 \degree$ is stirred continually and reaches $100 \degree$ after 10 min. At what time was it at $120 \degree$?
\textcolor{blue}{
\[
    T(t) = T_s +(T_0-T_s)e^{-kt} = 70+130e^{-kt}
\]
\[
    T(10) = 100 = 70+130e^{-k(10)} \text{ so } k= -\frac{\ln(\frac{3}{13})}{10} 
\]
\[
    T(t_1) = 120 = 70+130e^{-k(t_1)} \text{ so } t_1 = -\frac{\ln(\frac{5}{13})}{k} 
\]
\[
     t_1 = -\frac{\ln(\frac{5}{13})}{-\frac{\ln(\frac{3}{13})}{10}} =10\frac{\ln(\frac{5}{13})}{\ln(\frac{3}{13})} = \textcolor{red}{6.52 \text{ min}}
\]
}

