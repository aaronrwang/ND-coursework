\documentclass{article}
\usepackage[a4paper,margin=1in]{geometry}
\usepackage{amsmath, amsfonts, amssymb, amstext, mathtools,textcomp, xcolor, cancel}

\title{Homework 06}
\author{Aaron Wang}
\date{February 25 2025}

\begin{document}
\maketitle
\section{Monday 2/17}
\begin{large}
 \textbf{Section 9}   
\end{large}
\begin{enumerate}
    \item [7.] $y''- 4y' + 4y= 4$, $y_0=0$, $y_0'=-2$
\begin{gather*}
    L(y'')=p^2Y-py_0-y_0';\:
    L(y')=pY-y_0;\:
    L(y)=Y;\:
    L(4)=4/p
\end{gather*}
\begin{align*}
    y''- 4y' + 4y
    &=p^2Y-py_0-y_0'-4(pY-y_0)+4(Y)\\
    &=p^2Y-p(0)-(-2)-4(pY-(0))+4Y\\
    &=p^2Y+2-4pY+4Y\\
    &=p^2Y-4pY+4Y+2\\
    &=(p-2)^2Y+2
\end{align*}
\begin{gather*}
    (p-2)^2Y+2=\frac{4}{p}\\
    Y=\frac{4-2p}{p(p-2)^2}=\frac{-2}{p(p-2)}
\end{gather*}
Using partial fraction decomposition
\begin{gather*}
    Y=\frac{A}{p}+\frac{B}{p-2} \text{ s.t. } A(p-2)+Bp =  -2\\
    \implies (A+B)p-2A = -2 \text{ so } A  = 1\\
    \implies (1+B)p = 0 \text{ so } B  = -1\\
    Y=\frac{1}{p}-\frac{1}{p-2}
\end{gather*}
\begin{equation*}
    L^{-1}\Bigg(\frac{1}{p}\Bigg) 
    =1
    \tag{By \( L1 \)}
\end{equation*}
\begin{equation*}
    L^{-1}\Bigg(-\frac{1}{p-2}\Bigg) 
    = -L^{-1}\Bigg(\frac{1}{p+(-2)}\Bigg) 
    =-e^{-(-2)t}
    =-e^{2t}
    \tag{By \( L2 \)}\\
\end{equation*}
\textcolor{red}{\[
   y= 1-e^{2t}
\]}\newpage
    \item [9.] $y''+ 16y= 8 \cos{(4t)}$, $y_0=0$, $y_0'=8$
\begin{gather*}
    L(y'')=p^2Y-py_0-y_0';\:
    L(y')=pY-y_0;\:
    L(y)=Y;\:
    L(8 \cos{(4t)})=\frac{8p}{p^2+16}
\end{gather*}
\begin{align*}
    y''- 4y' + 4y
    &=p^2Y-py_0-y_0'+16(Y)\\
    &=p^2Y-p(0)-(8)+16Y\\
    &=p^2Y-8+16Y\\
    &=p^2Y+16Y-8\\
    &=(p^2+16)Y-8
\end{align*}
\begin{gather*}
    (p^2+16)Y-8=\frac{8p}{p^2+16}\\
    Y=\frac{8p}{(p^2+16)^2}+\frac{8}{p^2+16}
\end{gather*}
\begin{equation*}
    L^{-1}\Bigg(\frac{8p}{(p^2+16)^2}\Bigg) 
    = -L^{-1}\Bigg(\frac{2(4)p}{(p^2+(4)^2)^2}\Bigg) 
    =t \sin{((4)t)}
    =t \sin{(4t)}
    \tag{By \( L11 \)}
\end{equation*}
\begin{equation*}
    L^{-1}\Bigg(\frac{8}{p^2+16}\Bigg) 
    = 2L^{-1}\Bigg(\frac{(4)}{p^2+(4)^2}\Bigg) 
    =2\sin{((4)t)}
    =2\sin{(4t)}
    \tag{By \( L3 \)}\\
\end{equation*}
\textcolor{red}{\[
   y= t \sin{(4t)} + 2\sin{(4t)}
\]}\newpage
    \item [12.] $y''-y= e^{-t}-2te^{-t}$, $y_0=1$, $y_0'=2$
\begin{gather*}
    L(y'')=p^2Y-py_0-y_0';\:
    L(y')=pY-y_0;\:
    L(y)=Y;\:
    L(e^{-t}-2te^{-t})=\frac{1}{p+1}-\frac{2}{(p+1)^2}
\end{gather*}
\begin{align*}
    y''- y
    &=p^2Y-py_0-y_0'-(Y)\\
    &=p^2Y-p(1)-(2)-Y\\
    &=p^2Y-p-2-Y\\
    &=p^2Y-Y-p-2\\
    &=(p+1)(p-1)Y-p-2
\end{align*}
\begin{gather*}
    (p+1)(p-1)Y-p-2=\frac{1}{p+1}-\frac{2}{(p+1)^2}\\
    Y=\frac{\frac{1}{p+1}-\frac{2}{(p+1)^2}+p+2}{(p+1)(p-1)}
    =\frac{\frac{p-1}{(p+1)^2}+p+2}{(p+1)(p-1)}
    =\frac{1}{(p+1)^3}+\frac{p+2}{(p+1)(p-1)}
\end{gather*}
Using Partial Fraction Decomp:
\begin{gather*}
    \frac{p+2}{(p+1)(p-1)}=\frac{A}{p+1}+\frac{B}{p-1}\\
    \implies A(p-1)+B(p+1) = p+2
    \implies (A+B)p+(-A+B) = p+2\\
    \implies A+B = 1 \text { and }-A+B = 2
    \implies A+A+2 = 1 \implies A = -1/2 \implies B = 3/2\\
    \frac{p+2}{(p+1)(p-1)}=\frac{-1/2}{p+1}+\frac{3/2}{p-1}
\end{gather*}
\begin{gather*}
    Y=\frac{1}{(p+1)^3}+\frac{-1/2}{p+1}+\frac{3/2}{p-1}
\end{gather*}
\begin{equation*}
    L^{-1}\Bigg(\frac{1}{(p+1)^3}\Bigg) 
    = \frac{1}{2}L^{-1}\Bigg(\frac{(2)!}{(p+(1))^{(2)+1}}\Bigg) 
    =\frac{1}{2}t^{(2)} e^{-(1)t}
    =\frac{1}{2}t^2e^{-t}
    \tag{By \( L6 \)}
\end{equation*}
\begin{equation*}
    L^{-1}\Bigg(\frac{-1/2}{p+1}\Bigg) 
    = -\frac{1}{2}L^{-1}\Bigg(\frac{1}{p+(1)}\Bigg) 
    =-\frac{1}{2}e^{-(1)t}
    =-\frac{1}{2}e^{-t}
    \tag{By \( L2 \)}
\end{equation*}
\begin{equation*}
    L^{-1}\Bigg(\frac{3/2}{p-1}\Bigg) 
    = \frac{3}{2}L^{-1}\Bigg(\frac{1}{p+(-1)}\Bigg) 
    =\frac{3}{2}e^{-(-1)t}
    =\frac{3}{2}e^{t}
    \tag{By \( L2 \)}
\end{equation*}
\textcolor{red}{\[
   y= \frac{1}{2}t^2e^{-t}-\frac{1}{2}e^{-t}+\frac{3}{2}e^{t}
\]}\newpage
\end{enumerate}
\begin{large}
 \textbf{Section 10}   
\end{large}
Use the convolution integral to find the inverse transforms of:
\begin{enumerate}
    \item[3.]$\frac{p}{p^2-1}=\frac{p}{p^2-1}\cdot\frac{1}{p^2-1}$
\begin{equation*}
    L^{-1}\Bigg(\frac{p}{p^2-1}\Bigg) 
    = \cosh(t)
    \tag{By \( L10 \)}
\end{equation*}
\begin{equation*}
    L^{-1}\Bigg(\frac{1}{p^2-1}\Bigg) 
    = \sinh(t)
    \tag{By \( L9 \)}
\end{equation*}
\begin{align*}
    L^{-1}\Bigg( \frac{p}{p^2-1} \Bigg)
    & = \int_0^t \cosh (t-\tau) \sinh (\tau) d\tau \\
    & = \frac{1}{4}\int_0^t (e^{t-\tau}+e^{\tau-t}) (e^{\tau}-e^{-\tau}) d\tau\\
    & = \frac{1}{4}\int_0^t e^{t-\tau}e^{\tau}+e^{\tau-t}e^{\tau}-e^{t-\tau}e^{-\tau}-e^{\tau-t}e^{-\tau}d\tau \\
    & = \frac{1}{4}\int_0^t e^{t}+e^{2\tau-t}-e^{t-2\tau}-e^{-t}d\tau \\
    & = \frac{1}{4}\int_0^t e^{t}-e^{-t}+e^{2\tau-t}-e^{t-2\tau}d\tau \\
    & = \frac{1}{4}\Big(\tau e^{t}-\tau e^{-t}+\frac{e^{2\tau-t}}{2}+\frac{e^{t-2\tau}}{2}\Big)\Big|^t_0 \\
    & = \frac{1}{4}\Bigg(
    \Big(t e^{t}-t e^{-t}+\frac{e^{2t-t}}{2}+\frac{e^{t-2t}}{2}\Big)
    -\Big(0-0+\frac{e^{2(0)-t}}{2}+\frac{e^{t-2(0)}}{2}\Big)
    \Bigg) \\
    & = \frac{1}{4}\Big(t e^{t}-t e^{-t}
    \Big)\\
    & = \frac{t}{4}\Big(e^{t}-e^{-t}
    \Big)\\
    & = \textcolor{red}{\frac{1}{2}t \sinh{(t)}}
\end{align*}\newpage
    \item[4.]$\frac{1}{(p+a)(p+b)^2}$
\[
    \frac{1}{(p+a)(p+b)^2} = \frac{1}{(p+a)}\cdot\frac{1}{(p+b)^2}
\]
\begin{equation*}
    L^{-1}\Bigg(\frac{1}{(p+a)}\Bigg) 
    = e^{-at}
    \tag{By \( L2 \)}
\end{equation*}
\begin{equation*}
    L^{-1}\Bigg(\frac{1}{(p+b)^2}\Bigg) 
    = te^{-bt}
    \tag{By \( L6 \)}
\end{equation*}
\begin{align*}
    L^{-1}\Bigg( \frac{1}{(p+a)(p+b)^2} \Bigg)
    & = \int_0^t e^{-a(t-\tau)} \tau e^{-b\tau}d\tau \\
    & = \int_0^t \tau e^{-a(t-\tau)-b\tau}d\tau \\
    & = \int_0^t \tau e^{-at+(a-b)\tau}d\tau \\
    & = e^{-at}\int_0^t \tau e^{(a-b)\tau}d\tau \\
    & = e^{-at}\Bigg(\Big(\frac{\tau e^{(a-b)\tau}}{(a-b)}
    \Big)\Bigg|_0^t-\frac{1}{(a-b)} \int_0^t e^{(a-b)\tau}d\tau \Bigg)\\
    & = e^{-at}\Bigg(\frac{\tau e^{(a-b)\tau}}{(a-b)}
    -\frac{e^{(a-b)\tau}}{(a-b)^2}   \Bigg)\Bigg|_0^t\\
    & = e^{-at}\Bigg(
    \Big(\frac{t e^{(a-b)t}}{(a-b)}
    -\frac{e^{(a-b)t}}{(a-b)^2}\Big)
    -\Big(0
    -\frac{e^{(a-b)0}}{(a-b)^2}\Big)
    \Bigg)\\
    & = e^{-at}\Bigg(
    \frac{(a-b)t e^{(a-b)t}-e^{(a-b)t}+1}{(a-b)^2}
    \Bigg)\\
    & = \textcolor{red}{\frac{(a-b)t e^{-bt}-e^{-bt}+e^{-at}}{(a-b)^2}}\\
\end{align*}\newpage
    \item[5.]$\frac{p}{(p+a)(p+b)^2}$
\[
    \frac{1}{(p+a)(p+b)^2} = \frac{1}{(p+b)}\cdot\frac{p}{(p+a)(p+b)}
\]
\begin{equation*}
    L^{-1}\Bigg(\frac{1}{(p+b)}\Bigg) 
    = e^{-bt}
    \tag{By \( L2 \)}
\end{equation*}
\begin{equation*}
    L^{-1}\Bigg(\frac{p}{(p+a)(p+b)}\Bigg) 
    = \frac{ae^{-at}-be^{-bt}}{a-b}
    \tag{By \( L8 \)}
\end{equation*}
\begin{align*}
    L^{-1}\Bigg( \frac{p}{(p+a)(p+b)^2} \Bigg)
    & = \int_0^t e^{-b(t-\tau)} \frac{ae^{-a\tau}-be^{-b\tau}}{a-b}d\tau \\
    & = \frac{e^{-bt}}{a-b}\int_0^te^{b\tau} (ae^{-a\tau}-be^{-b\tau})d\tau\\
    & = \frac{e^{-bt}}{a-b}\int_0^t ae^{(b-a)\tau}-be^{(b-b)\tau}d\tau\\
    & = \frac{e^{-bt}}{a-b}\int_0^t ae^{(b-a)\tau}-bd\tau\\
    & = \frac{e^{-bt}}{a-b}\Big( \frac{ae^{(b-a)\tau}}{(b-a)}-b\tau\Big)\Big|_0^t\\
    & = \frac{e^{-bt}}{a-b}\Bigg( 
    \Big(\frac{ae^{(b-a)t}}{(b-a)}-bt\Big)
    -\Big(\frac{ae^{(b-a)(0)}}{(b-a)}-b(0)\Big)
    \Bigg)\\
    & = \frac{e^{-bt}}{a-b}\Bigg( 
    \frac{ae^{(b-a)t}}{(b-a)}-bt
    -\frac{a}{(b-a)}
    \Bigg)\\
    & = \frac{e^{-bt}}{a-b}\Bigg( 
    -\frac{ae^{(b-a)t}}{a-b}+\frac{(b-a)bt}{a-b}
    +\frac{a}{a-b}
    \Bigg)\\
    & = \frac{e^{-bt}}{(a-b)^2}\Bigg( 
    -ae^{(b-a)t}+(b-a)bt
    +a
    \Bigg)\\
    & = \frac{1}{(a-b)^2}\Bigg( 
    -ae^{-bt}e^{(b-a)t}+(b-a)bte^{-bt}
    +ae^{-bt}
    \Bigg)\\
    & = \frac{1}{(a-b)^2}\Bigg( 
    -ae^{-at}+(b-a)bte^{-bt}
    +ae^{-bt}
    \Bigg)\\
    & = \textcolor{red}{\frac{-ae^{-at}+(b-a)bte^{-bt}
    +ae^{-bt}}{(a-b)^2}}
\end{align*}\newpage
\end{enumerate}
\newpage
\section{Wednesday 2/19}
\begin{large}
 \textbf{Section 10}   
\end{large}
Use the convolution integral to find the inverse transforms of:
\begin{enumerate}
    \item[9.]$\frac{2}{p^3(p+2)}$
\[
    \frac{2}{p^3(p+2)} = \frac{2}{p^3}\cdot\frac{1}{p+2}
\]
\begin{equation*}
    L^{-1}\Bigg(\frac{2}{p^3}\Bigg) 
    = t^2
    \tag{By \( L5 \)}
\end{equation*}
\begin{equation*}
    L^{-1}\Bigg(\frac{1}{p+2}\Bigg) 
    = e^{-2t}
    \tag{By \( L2 \)}
\end{equation*}
\begin{align*}
    L^{-1}\Bigg( \frac{2}{p^3(p+2)} \Bigg)
    & = \int_0^t \tau^2e^{-2(t-\tau)} d\tau \\
    & =e^{-2t} \int_0^t \tau^2e^{2\tau} d\tau \\
    & =e^{-2t}\Bigg(\frac{\tau^2e^{2\tau}}{2}\Big|^t_0 -\int_0^t \tau e^{2\tau} d\tau \Bigg)\\
    & =e^{-2t}\Bigg(\Big(\frac{\tau^2e^{2\tau}}{2}-\frac{\tau e^{2\tau}}{2}\Big)\Big|^t_0 +\frac{1}{2}\int_0^t e^{2\tau} d\tau \Bigg)\\
    & =e^{-2t}\Bigg(\frac{\tau^2e^{2\tau}}{2}-\frac{\tau e^{2\tau}}{2} +\frac{e^{2\tau}}{4} \Bigg)\Bigg|^t_0\\
    & =e^{-2t}\Bigg(
    \Big(\frac{t^2e^{2t}}{2}-\frac{t e^{2t}}{2} +\frac{e^{2t}}{4} \Big)
    -\Big(\frac{(0)^2e^{2(0)}}{2}-\frac{(0) e^{2(0)}}{2} +\frac{e^{2(0)}}{4} \Big)
    \Bigg)\\
    & =e^{-2t}\Bigg(
    \frac{t^2e^{2t}}{2}-\frac{t e^{2t}}{2} +\frac{e^{2t}}{4} 
    -\frac{1}{4} \Bigg)\\
    & =\textcolor{red}{
    \frac{t^2}{2}-\frac{t}{2} +\frac{1}{4} 
    -\frac{e^{-2t}}{4}}
\end{align*}\newpage
    \item[10.]$\frac{1}{p(p^2+a^2)^2}$
\[
    \frac{1}{p(p^2+a^2)^2} = \frac{1}{a^3} \cdot \frac{a^2}{p(p^2+a^2)} \cdot \frac{a}{p^2+a^2}
\]
\begin{equation*}
    L^{-1}\Bigg(\frac{a^2}{p(p^2+a^2)} \Bigg) 
    = 1-\cos(at)
    \tag{By \( L15 \)}
\end{equation*}
\begin{equation*}
    L^{-1}\Bigg(\frac{a}{p^2+a^2}\Bigg) 
    = \sin(at)
    \tag{By \( L3 \)}
\end{equation*}
\begin{align*}
    L^{-1}\Bigg( \frac{1}{p(p^2+a^2)^2} \Bigg)
    & = \frac{1}{a^3}\int_0^t \big(1-\cos(a(t-\tau))\big)\sin(a\tau)d\tau \\
    & = \frac{1}{a^3}\int_0^t \sin(a\tau)-\sin(a\tau)\cos(at-a\tau)d\tau \\
    & = \frac{1}{a^3}\Bigg(\int_0^t \sin(a\tau)d\tau-\int_0^t\sin(a\tau)\cos(at-a\tau)d\tau\Bigg) \\
\end{align*}
\[
    \int_0^t \sin(a\tau)d\tau =-\frac{\cos(a\tau)}{a}\Big|^t_0=-\Big(\frac{\cos(at)}{a}-\frac{\cos(a(0))}{a}\Big)=\frac{1}{a}-\frac{\cos(at)}{a}
\]
\begin{align*}
   \int_0^t\sin(a\tau)\cos(at-a\tau)d\tau
   & = \int_0^t \frac{1}{2}
   \Bigg[
   \sin\Big(a\tau+(at-a\tau)\Big)
   + \sin\Big(a\tau-(at-a\tau)\Big)
   \Bigg] d\tau\\
   & = \frac{1}{2}\int_0^t 
   \sin(at)
   + \sin(2a\tau-at) d\tau\\
   & = \frac{1}{2}\Big(
   \tau\sin(at)- \frac{1}{2a}\cos(2a\tau-at)
   \Big)\Big|^t_0\\
   & = \frac{1}{2}\tau\sin(at)- \frac{1}{4a}\cos(2a\tau-at)\Big|^t_0\\
   & = \Big(\frac{1}{2}t\sin(at)- \frac{1}{4a}\cos(2at-at)\Big)
   -\Big(\frac{1}{2}(0)\sin(at)- \frac{1}{4a}\cos(2a(0)-at)\Big)\\
   & = \frac{1}{2}t\sin(at)- \frac{1}{4a}\cos(at)
   -0+ \frac{1}{4a}\cos(-at)\\
   & = \frac{1}{2}t\sin(at)- \frac{1}{4a}\cos(at) + \frac{1}{4a}\cos(at)\\
   & = \frac{1}{2}t\sin(at)\\
\end{align*}
\begin{align*}
    L^{-1}\Bigg( \frac{1}{p(p^2+a^2)^2} \Bigg)
    & = \frac{1}{a^3}\Bigg(\frac{1}{a}-\frac{\cos(at)}{a}-\frac{t\sin(at)}{2}\Bigg) \\
    & = \textcolor{red}{\frac{1}{a^4}-\frac{\cos(at)}{a^4}-\frac{t\sin(at)}{2a^3}} \\
\end{align*}\newpage
    \item[11.]$\frac{p}{(p^2+a^2)(p^2+b^2)}$
\[
    \frac{p}{(p^2+a^2)(p^2+b^2)} = \frac{1}{a}\cdot\frac{a}{p^2+a^2}\cdot\frac{p}{p^2+b^2}
\]
\begin{equation*}
    L^{-1}\Bigg(\frac{a}{p^2+a^2}\Bigg) 
    = \sin(at)
    \tag{By \( L3 \)}
\end{equation*}
\begin{equation*}
    L^{-1}\Bigg(\frac{p}{p^2+b^2}\Bigg) 
    = \cos(bt)
    \tag{By \( L4 \)}
\end{equation*}
\begin{align*}
    L^{-1}\Bigg( \frac{p}{(p^2+a^2)(p^2+b^2)} \Bigg)
    & = \frac{1}{a}\int_0^t \sin(a\tau)\cos(b(t-\tau))d\tau\\
     & = \frac{1}{a}\int_0^t \sin(a\tau)\cos(bt-b\tau))d\tau\\
    & = \frac{1}{a}\int_0^t \frac{1}{2}\Bigg[ \sin(a\tau+(bt-b\tau))+\sin(a\tau-(bt-b\tau))
    \Bigg]d\tau\\
    & = \frac{1}{2a} \int_0^t \sin((a-b)\tau+bt)+\sin((a+b)\tau-bt)d\tau\\
    & = \frac{1}{2a} \Bigg( \frac{-\cos((a-b)\tau+bt)}{a-b}+\frac{-\cos((a+b)\tau-bt)}{a+b}\Bigg)\Bigg|^t_0\\
    & =\Bigg(\frac{\cos((a-b)\tau+bt)}{2a(b-a)}-\frac{\cos((a+b)\tau-bt)}{2a(a+b)}\Bigg)\Bigg|^t_0\\
    & =\Bigg(\frac{\cos((a-b)t+bt)}{2a(b-a)}-\frac{\cos((a+b)t-bt)}{2a(a+b)}\Bigg)\\
    & -\Bigg(\frac{\cos((a-b)0+bt)}{2a(b-a)}-\frac{\cos((a+b)0-bt)}{2a(a+b)}\Bigg)\\
    & =\Bigg(\frac{\cos(at)}{2a(b-a)}-\frac{\cos(at)}{2a(a+b)}\Bigg)
    -\Bigg(\frac{\cos(bt)}{2a(b-a)}-\frac{\cos(bt)}{2a(a+b)}\Bigg)\\
    & = \frac{((a+b)-(b-a))\cos(at)}{2a(b-a)(a+b)}-\Bigg(\frac{\cos(bt)}{2a(b-a)}-\frac{\cos(bt)}{2a(a+b)}\Bigg)\\
    & = \frac{2a\cos(at)}{2a(b-a)(a+b)}-\Bigg(\frac{\cos(bt)}{2a(b-a)}-\frac{\cos(bt)}{2a(a+b)}\Bigg)\\
    & = \frac{2a\cos(at)}{2a(b-a)(a+b)}-\frac{((a+b)-(b-a))\cos(bt)}{2a(b-a)(a+b)}\\
    & = \frac{2a\cos(at)}{2a(b-a)(a+b)}-\frac{2a\cos(bt)}{2a(b-a)(a+b)}\\
    & = \frac{\cos(at)}{(b-a)(a+b)}-\frac{\cos(bt)}{(b-a)(a+b)}\\
    & = \textcolor{red}{\frac{\cos(at)}{b^2-a^2}-\frac{\cos(bt)}{b^2-a^2}}\\
\end{align*}\newpage
    \item[14.] $y'' + 5y'+ 6y= e^{-2t}, y_0 = y_0' = 0$
\begin{gather*}
    L(y'')=p^2Y-py_0-y_0';\:
    L(y')=pY-y_0;\:
    L(y)=Y;\:\\
    L(y'' + 5y'+ 6y)=(p^2Y-py_0-y_0')+5(pY-y_0)+6(Y)\\
    L(y'' + 5y'+ 6y)=p^2Y+5pY+6Y=(p+2)(p+3)Y\\
    (p+2)(p+3)Y=L(e^{2t}) \implies Y=\frac{1}{(p+2)(p+3)}\cdot L(e^{2t})\\
\end{gather*}
\begin{equation*}
    L^{-1}\Bigg(\frac{1}{(p+2)(p+3)} \Bigg) 
    = \frac{e^{-(3)t}-e^{-(2)t}}{2-(3)}=e^{-2t}-e^{-3t}
    \tag{By \( L7 \)}
\end{equation*}
\begin{align*}
    y 
    & = \int^t_0 (e^{-2\tau}-e^{-3\tau})e^{-2(t-\tau)}d\tau\\
    & = e^{-2t}\int^t_0 (e^{-2\tau}-e^{-3\tau})e^{2\tau}d\tau\\
    & = e^{-2t}\int^t_0 1-e^{-\tau}d\tau\\
    & = e^{-2t}(\tau+e^{-\tau})\Big|^t_0\\
    & = e^{-2t}\Big((t+e^{-t})-(0+e^{-0}\Big)\\
    & = e^{-2t}(t+e^{-t}-1)\\
    & = \textcolor{red}{te^{-2t}+e^{-3t}-e^{-2t}}\\
\end{align*}\newpage
    \item[15.] $y'' + 3y'- 4y= e^{3t}, y_0 = y_0' = 0$
\begin{gather*}
    L(y'')=p^2Y-py_0-y_0';\:
    L(y')=pY-y_0;\:
    L(y)=Y;\:\\
    L(y'' + 3y'- 4y)=(p^2Y-py_0-y_0')+3(pY-y_0)-4(Y)\\
    L(y'' + 3y'- 4y)=p^2Y+3pY-4Y=(p+4)(p-1)Y\\
    (p+4)(p-1)Y=L(e^{3t}) \implies Y=\frac{1}{(p+4)(p-1)}\cdot L(e^{3t})\\
\end{gather*}
\begin{equation*}
    L^{-1}\Bigg(\frac{1}{(p+4)(p-1)} \Bigg) 
    = \frac{e^{-(-1)t}-e^{-(4)t}}{4-(-1)}=\frac{e^{t}-e^{-4t}}{5}
    \tag{By \( L7 \)}
\end{equation*}
\begin{align*}
    y 
    & = \int^t_0 \frac{e^{\tau}-e^{-4\tau}}{5}e^{3(t-\tau)}d\tau\\
    & = \frac{e^{3t}}{5}\int^t_0 (e^{\tau}-e^{-4\tau})e^{-3\tau}d\tau\\
    & = \frac{e^{3t}}{5}\int^t_0 e^{-2\tau}-e^{-7\tau}d\tau\\
    & = \frac{e^{3t}}{5}\Big( \frac{1}{7}e^{-7\tau}-\frac{1}{2}e^{-2\tau}\Big)\Big|^t_0\\
    & = \frac{e^{3t}}{5}
    \Bigg(
    \Big( \frac{1}{7}e^{-7t}-\frac{1}{2}e^{-2t}\Big)
    -\Big( \frac{1}{7}e^{-7(0)}-\frac{1}{2}e^{-2(0)}\Big)
    \Bigg)\\
    & = \frac{e^{3t}}{5}
    \Big(\frac{1}{7}e^{-7t}-\frac{1}{2}e^{-2t} + \frac{5}{14}\Big)\\
    & = \textcolor{red}{\frac{1}{35}e^{-4t}-\frac{1}{10}e^{t} + \frac{1}{14}e^{3t}}
\end{align*}\newpage
\end{enumerate}



\end{document}
% sec 9: 7,9,12
% sec 10: 3,4,5