\documentclass{article}
\usepackage{amsmath, amsfonts, amssymb, amstext, mathtools}
\usepackage{textcomp}
\usepackage[left=1in, right=1.5in]{geometry}
\usepackage{xcolor}

\newcommand{\Emptyset}{\varnothing}
\newcommand{\notsubseteq}{\mathrel{\not\subseteq}}
\newcommand{\union}{\cup}
\newcommand{\intersect}{\cap}
\newcommand{\defeq}{\coloneqq}
\newcommand{\naturals}{\mathbb{N}}
\newcommand{\s}{\mathbb{S}}
\newcommand{\power}{\mathbb{P}}
\newcommand{\expect}{E}
\newcommand{\var}{Var}

\title{Homework 3}
\author{Aaron Wang}
\date{September 30 2024}

\begin{document}
\maketitle
    \begin{enumerate}
        \item A single fair die is tossed once. Let $Y$ be the number facing up. Find the expected value and variance of $Y$.
        \begin{enumerate}
            \item 
        
        \begin{equation*}
            \expect(Y) = \sum\limits_y y\cdot p(y)
        \end{equation*}
        \begin{equation*}
            = 
            1\cdot p(1) +
            2\cdot p(2) +
            3\cdot p(3) +
            4\cdot p(4) +
            5\cdot p(5) +
            6\cdot p(6)
        \end{equation*}
        \begin{equation*}
            = 
            1\cdot \frac{1}{6} +
            2\cdot \frac{1}{6} +
            3\cdot \frac{1}{6} +
            4\cdot \frac{1}{6} +
            5\cdot \frac{1}{6} +
            6\cdot \frac{1}{6}
        \end{equation*}
        \begin{equation*}
            = 
            \frac{1}{6} +
            \frac{2}{6} +
            \frac{3}{6} +
            \frac{4}{6} +
            \frac{5}{6} +
            \frac{6}{6} = \textcolor{red}{3.5}
        \end{equation*}
            \item
        \begin{equation*}
            \var(Y) = \expect(Y^2) - \expect(Y)^2
        \end{equation*}
        \begin{equation*}
            = 
            1\cdot p(1) +
            4\cdot p(2) +
            9\cdot p(3) +
            16\cdot p(4) +
            25\cdot p(5) +
            36\cdot p(6) - 
            3.5^2
        \end{equation*}
        \begin{equation*}
            = 
            1\cdot \frac{1}{6} +
            4\cdot \frac{1}{6} +
            9\cdot \frac{1}{6} +
            16\cdot \frac{1}{6} +
            25\cdot \frac{1}{6} +
            36\cdot \frac{1}{6} - 
            12.25
        \end{equation*}
        \begin{equation*}
            = 
            \frac{1}{6} +
            \frac{4}{6} +
            \frac{9}{6} +
            \frac{16}{6} +
            \frac{25}{6} +
            \frac{36}{6} - 
            12.25
        \end{equation*}
        \begin{equation*}
            = 15.17-12.25=\textcolor{red}{2.92}
        \end{equation*}
        \end{enumerate}
\pagebreak
        \item A multiple-choice test consists of 20 items, each with four choices. A student is able to eliminate one of the choices on each question as incorrect and chooses randomly from the remaining three choices. A passing grade is 18 items or more correct.
        \begin{enumerate}
            \item What is the probability that the student passes\\\\
                There are $n=20$ questions. There are 4 answer choices. Since one can be eliminated, there are essentially only three answer choices. Thus, the probabilty of Success is $p=\frac{1}{3}$ and the probability of failure is $q=\frac{2}{3}$. Notice that this experiment is a binomial experiment as it follows the 5 properties. The probability that a student passes is $p(\geq 18)$.
                \begin{equation*}
                    p(Y\geq 18) = p(Y=18)+p(Y=19)+p(Y=20)
                \end{equation*}
                \begin{equation*}
                    p(y)=\binom{n}{y}p^yq^{n-y}
                \end{equation*}
                \begin{equation*}
                    p(18)=\binom{20}{18}\frac{1}{3}^{18}\frac{2}{3}^{20-18}=2.18\cdot10^{-7}
                \end{equation*}
                \begin{equation*}
                    p(19)=\binom{20}{19}\frac{1}{3}^{19}\frac{2}{3}^{20-19}=1.15\cdot10^{-8}
                \end{equation*}
                \begin{equation*}
                    p(20)=\binom{20}{20}\frac{1}{3}^{20}\frac{2}{3}^{20-20}=2.87\cdot10^{-10}
                \end{equation*}
                \begin{equation*}
                    p(Y\geq 18) = \textcolor{red}{2.30\cdot10^{-7}}
                \end{equation*}
                
            \item Answer the question in part (a) again, assuming that the student can eliminate
two of the choices on each question.\\\\
                We can follow the same logic as before except this time success is $p=\frac{1}{2}$ and failure is $q=\frac{1}{2}$.
                \begin{equation*}
                    p(Y\geq 18) = p(Y=18)+p(Y=19)+p(Y=20)
                \end{equation*}
                \begin{equation*}
                    p(y)=\binom{n}{y}p^yq^{n-y}
                \end{equation*}
                \begin{equation*}
                    p(18)=\binom{20}{18}\frac{1}{2}^{18}\frac{1}{2}^{20-18}=1.81\cdot10^{-4}
                \end{equation*}
                \begin{equation*}
                    p(19)=\binom{20}{19}\frac{1}{2}^{19}\frac{1}{2}^{20-19}=1.91\cdot10^{-5}
                \end{equation*}
                \begin{equation*}
                    p(20)=\binom{20}{20}\frac{1}{2}^{20}\frac{1}{2}^{20-20}=9.54\cdot10^{-7}
                \end{equation*}
                \begin{equation*}
                    p(Y\geq 18) = \textcolor{red}{2.01\cdot10^{-4}}
                \end{equation*}
        \end{enumerate}
\pagebreak
        \item Suppose that in a sequence of independent Bernoulli trials, each with probability
of success p, the number of failures up to the first success is counted. 
        \begin{enumerate}
            \item What is the probability mass function for this random variable?
            \begin{equation*}
                p(X=k)=(1-p)^kp
            \end{equation*}
            \item Continuing with Part (a), find the probability mass function for the number of failures up to the r-th success.
            \begin{equation*}
                p(Z=k)=\binom{k+r-1}{r-1}p^r(1-p)^{k}
            \end{equation*}
            This is modeled by the Negative Binomial distribution such that k is the number of failures up to the r-th success (not the total number of trials).
        \end{enumerate}
        \item[5.] In a gambling game a person draws a single card from an ordinary 52-card playing
deck. A person is paid $\$15$ for drawing a jack or a queen and $\$5$ for drawing a king or an ace. A person who draws any other card pays $\$4$. If a person plays this game, what is the expected gain?



        P(J,Q)$=\frac{8}{52}=\frac{2}{13}$
        
        P(K,A)$=\frac{8}{52}=\frac{2}{13}$
        
        P(2,3,4,5,6,7,8,9,10)$=\frac{36}{52}=\frac{9}{13}$
        \begin{equation*}
            \expect(Y) = \sum\limits_y y\cdot p(y)
        \end{equation*}
        \begin{equation*}
            \expect(Y) = 15 \cdot p(J,Q) + 5 \cdot p(K,A) + (-4) \cdot p(2,3,4,5,6,7,8,9,10)
        \end{equation*}
        \begin{equation*}
            =15 \cdot \frac{2}{13} + 5 \cdot \frac{2}{13} + (-4) \cdot \frac{9}{13}
        \end{equation*}
        \begin{equation*}
            \frac{30}{13} + \frac{10}{13} - \frac{36}{13}=\textcolor{red}{\frac{4}{13}}
        \end{equation*}
        \item[6.] An oil prospector will drill a succession of holes in a given area to find a productive well. The probability that he is successful on a given trial is 0.2.
        \begin{enumerate}
            \item What is the probability that the third hole drilled is the first to yield a productive well?\\\\
            We have a geometric random variable Y and we want to find $p(Y=3)$.
            \begin{equation*}
                p(y)=q^{y-1} \cdot p=(1-p)^{y-1} \cdot p
            \end{equation*}
            \begin{equation*}
                p(3)=(1-0.2)^{3-1} \cdot 0.2=\textcolor{red}{0.128}
            \end{equation*}
            \item If the prospector can afford to drill at most ten wells, what is the probability that he will fail to find a productive well?\\\\
            To find the probability of success, let's consider Y as a variable of a binomial experiment. Consider p = 0.2 (we want failure), q=0.8, we have 10 trials, and want 0 to successes.
            \begin{equation*}
                p(0)=\binom{10}{0}\cdot0.2^{0}\cdot0.8^{10-0}=\textcolor{red}{0.107}
            \end{equation*}
            
        \end{enumerate}
    \end{enumerate}
\end{document}