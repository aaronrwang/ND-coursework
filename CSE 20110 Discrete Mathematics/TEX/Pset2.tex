\documentclass{article}
\usepackage{amsmath}
\usepackage{amsthm}
\usepackage{mathtools}
\usepackage[left=1.5in, right=2in]{geometry}
\usepackage{xcolor}
\renewcommand\qedsymbol{Q.E.D.}

\title{Problem Set 2}
\author{Aaron Wang}
\date{February 4 2024}
\newcommand{\negland}{\mathrel{\neg\!\!\land}}

\begin{document}

\maketitle
%q1%
\begin {enumerate}
    \item Consider the following proof of $p \rightarrow (q \rightarrow r) \equiv (p \rightarrow q) \rightarrow r$.

      \vspace{2ex}
      \begin{proof}
        Observe the following chain of reasoning.
        \begin{align*}
          p \rightarrow (q \rightarrow r)
            &\equiv p \lor \neg(q \rightarrow r)
              &\quad
              &\text{by \emph{conditional disintegration}} 
              & (1)
              \\
            &\equiv p \lor \neg(q \lor \neg r)
              &\quad
              &\text{by \emph{conditional disintegration}} & (2)
              \\
            &\equiv p \lor \neg q \lor \neg r
              &\quad
              &\text{by \emph{associativity}} 
              & (3)
              \\
            % &\equiv (p \lor \neg q) \lor \neg r
              % &\quad
              % &\text{by \emph{associativity}} \\
            &\equiv (p \lor \neg q) \lor \neg r
              &\quad
              &\text{by \emph{associativity}} 
              & (4)
              \\
            &\equiv (p \rightarrow q) \lor \neg r
              &\quad
              &\text{by \emph{conditional disintegration}} & (5)
              \\
            &\equiv (p \rightarrow q) \rightarrow r
              &\quad
              &\text{by \emph{conditional disintegration}}
              & (6)
        \end{align*}
        
        Therefore, $p \rightarrow (q \rightarrow r) \equiv (p \rightarrow q) \rightarrow r$.
        \end{proof}
        
        Find all of the mistakes, if any, in this proof, and explain why.

        \textcolor{red}{p,q and r are not declared as propositions.}
        
        \textcolor{red}{In lines 1, 2, 5, and 6, conditional disintegration is incorrectly used}. The axiom states ``$p\rightarrow q\equiv\neg p\lor q$" yet the proof incorrectly uses ``$p\rightarrow q\equiv p\lor\neg q$"
        
        \textcolor{red}{From (2) to (3) the proof incorrectly uses associativity}. First, it incorrectly distributes $\neg$ and then it gets rid of the parenthesis, two things that should not happen.
\newgeometry{left=0.75in, right=1.25in}
\pagebreak  
%q2%
    \item
        Prove the claims below \emph{without truth tables} for all propositions $p, q, r$.

    \begin{enumerate}
%2a%
      \item
        $p \rightarrow q \equiv \neg q \rightarrow \neg p$.
        
        \vspace{2ex}
        \begin{proof}
        Let p and q be propositions. Observe the following chain of reasoning.
        \begin{align*}
          p \rightarrow q 
            &\equiv \neg p \lor q
              &\quad
              &\text{by \emph{conditional disintegration}} 
              \\
            &\equiv q \lor \neg p
              &\quad
              &\text{by \emph{commutativity}} 
              \\
            &\equiv \neg (\neg q) \lor \neg p
              &\quad
              &\text{by \emph{double negation}} 
              \\
            &\equiv \neg q \rightarrow \neg p
              &\quad
              &\text{by \emph{conditional disintegration}} 
        \end{align*}
        Therefore, $p \rightarrow q \equiv \neg q \rightarrow \neg p$.
        \end{proof}
\hrulefill
 
%2b%       
      \item
        $(p \land (p \rightarrow q)) \rightarrow q$ is a tautology.

        \vspace{2ex}
        \begin{proof}
        Let p and q be propositions. Observe the following chain of reasoning.
        \begin{align*}
          (p \land (p \rightarrow q)) \rightarrow q
            &\equiv (p \land (\neg p \lor q)) \rightarrow q
              &\quad
              &\text{by \emph{conditional disintegration}}
              \\
            &\equiv ((p \land \neg p) \lor (p \land q)) \rightarrow q
              &\quad
              &\text{by \emph{distributivity}}
              \\
            &\equiv (\bot \lor (p \land q)) \rightarrow q
              &\quad
              &\text{by \emph{complement}}
              \\
            &\equiv (p \land q) \rightarrow q
              &\quad
              &\text{by \emph{identity}}
              \\
            &\equiv \neg(p \land q) \lor q
              &\quad
              &\text{by \emph{conditional disintegration}}
              \\
            &\equiv (\neg p \lor \neg q) \lor q
              &\quad
              &\text{by \emph{de morgan's laws}}
              \\
            &\equiv \neg p \lor (\neg q \lor q)
              &\quad
              &\text{by \emph{associativity}}
              \\
            &\equiv \neg p \lor \top
              &\quad
              &\text{by \emph{complement}}
              \\
            &\equiv \top \lor \neg p
              &\quad
              &\text{by \emph{commutativity}}
              \\
            &\equiv \top
              &\quad
              &\text{by \emph{domination}}
              \\
        \end{align*}
        Therefore, $(p \land (p \rightarrow q)) \rightarrow q$ is a tautology.
        \end{proof}
\pagebreak
%2c%
      \item
        $(\neg q \land (p \rightarrow q)) \rightarrow \neg p$ is a tautology.

        \vspace{2ex}
        \begin{proof}
        Let p and q be propositions. Observe the following chain of reasoning.
        \begin{align*}
          (\neg q \land (p \rightarrow q)) \rightarrow \neg p
            &\equiv (\neg q \land (\neg p \lor q)) \rightarrow \neg p
              &\quad
              &\text{by \emph{conditional disintegration}}
              \\
            &\equiv ((\neg q \land \neg p) \lor (\neg q \land  q)) \rightarrow \neg p
              &\quad
              &\text{by \emph{distributivity}}
              \\
            &\equiv ((\neg q \land \neg p) \lor \bot) \rightarrow \neg p
              &\quad
              &\text{by \emph{complement}}
              \\
            &\equiv (\bot \lor (\neg q \land \neg p)) \rightarrow \neg p
              &\quad
              &\text{by \emph{commutativity}}
              \\
            &\equiv (\neg q \land \neg p) \rightarrow \neg p
              &\quad
              &\text{by \emph{identity}}
              \\
            &\equiv \neg(\neg q \land \neg p) \lor \neg p
              &\quad
              &\text{by \emph{conditional disintegration}}
              \\
            &\equiv (q \lor p) \lor \neg p
              &\quad
              &\text{by \emph{de morgans laws}}
              \\
            &\equiv q \lor (p \lor \neg p)
              &\quad
              &\text{by \emph{associativity}}
              \\
            &\equiv q \lor (\neg p \lor p)
              &\quad
              &\text{by \emph{commutativity}}
              \\
            &\equiv q \lor \top
              &\quad
              &\text{by \emph{complement}}
              \\
            &\equiv \top \lor q
              &\quad
              &\text{by \emph{commutativity}}
              \\
            &\equiv \top
              &\quad
              &\text{by \emph{domination}}
              \\
        \end{align*}
        Therefore, $(\neg q \land (p \rightarrow q)) \rightarrow \neg p$ is a tautology.
        \end{proof}
        
\hrulefill

%2d%
      \item
        $(p \rightarrow q) \rightarrow ((p \rightarrow \neg q) \rightarrow \neg p)$ is a tautology.

        \vspace{2ex}
        \begin{proof}
        Let p and q be propositions. Observe the following chain of reasoning.
        \begin{align*}
          (p \rightarrow q) \rightarrow ((p \rightarrow \neg q) \rightarrow \neg p)
            &\equiv(p \rightarrow q) \rightarrow ((\neg p \lor \neg q) \rightarrow \neg p)
              &\quad
              &\text{by \emph{conditional disintegration}}
              \\
            &\equiv(p \rightarrow q) \rightarrow (\neg(\neg p \lor \neg q) \lor \neg p)
              &\quad
              &\text{by \emph{conditional disintegration}}
              \\
            &\equiv(p \rightarrow q) \rightarrow (\neg(\neg p)\land \neg(\neg q)) \lor \neg p)
              &\quad
              &\text{by \emph{de morgans laws}}
              \\
            &\equiv(p \rightarrow q) \rightarrow ((p \land q) \lor \neg p)
              &\quad
              &\text{by \emph{double negation}}
              \\
            &\equiv(p \rightarrow q) \rightarrow (\neg p \lor(p \land q))
              &\quad
              &\text{by \emph{double negation}}
              \\
            &\equiv(p \rightarrow q) \rightarrow ((\neg p \lor p) \land (\neg p \lor q))
              &\quad
              &\text{by \emph{distributivity}}
              \\
            &\equiv(p \rightarrow q) \rightarrow (\top \land (\neg p \lor q))
              &\quad
              &\text{by \emph{complement}}
              \\
            &\equiv(p \rightarrow q) \rightarrow (\neg p \lor q)
              &\quad
              &\text{by \emph{identity}}
              \\
            &\equiv(\neg p \lor q) \rightarrow (\neg p \lor q)
              &\quad
              &\text{by \emph{conditional disintegration}}
              \\
            &\equiv \neg(\neg p \lor q) \lor (\neg p \lor q)
              &\quad
              &\text{by \emph{conditional disintegration}}
              \\
            &\equiv \top
              &\quad
              &\text{by \emph{complement}}
              \\
        \end{align*}
        Therefore, $(p \rightarrow q) \rightarrow ((p \rightarrow \neg q) \rightarrow \neg p)$ is a tautology.
        \end{proof}
    \end{enumerate}
\pagebreak
%q3%
    \item In this problem, we will progressively establish that the alternative axioms Hilbert proposed are all tautologies \emph{without truth tables}.
    Here, the variables $p$, $q$, and $r$ all represent arbitrary propositions.
    \begin{enumerate}
%3a%
      \item
        Show $p \rightarrow p$ is a tautology.

        \vspace{2ex}
        \begin{proof}
        Let p be a proposition. Observe the following chain of reasoning.
        \begin{align*}
          p \rightarrow p
           &\equiv \neg p \lor p
            &\quad
            &\text{by \emph{conditional disintegration}}
            \\
           &\equiv \top
            &\quad
            &\text{by \emph{complement}}
            \\
        \end{align*}
        Therefore, $p \rightarrow p$ is a tautology.
        \end{proof}

\hrulefill

%3b%
      \item
        Show $(p \rightarrow q) \rightarrow (\neg q \rightarrow \neg p)$ is a tautology.

        \vspace{2ex}
        \begin{proof}
        Let p and q be propositions. Observe the following chain of reasoning.
        \begin{align*}
        (p \rightarrow q) \rightarrow (\neg q \rightarrow \neg p)
          &\equiv(\neg p \lor q) \rightarrow  (\neg(\neg q) \lor \neg p)
            &\quad
            &\text{by \emph{conditional disintegration}}\times2
            \\
          &\equiv(\neg p \lor q) \rightarrow  (q \lor \neg p)
            &\quad
            &\text{by \emph{double negation}}
            \\
          &\equiv(\neg p \lor q) \rightarrow  (\neg p \lor q)
            &\quad
            &\text{by \emph{commutativity}}
            \\
          &\equiv \neg(\neg p \lor q) \lor  (\neg p \lor q)
            &\quad
            &\text{by \emph{conditional disintegration}}
            \\
          &\equiv \top
            &\quad
            &\text{by \emph{complement}}
            \\
        \end{align*}
        Therefore, $(p \rightarrow q) \rightarrow  (\neg q \rightarrow \neg p)$ is a tautology.
        \end{proof}
\pagebreak
%3c%
      \item
        Show $p \rightarrow (q \rightarrow p)$ is a tautology.

        \vspace{2ex}
        \begin{proof}
        Let p and q be propositions. Observe the following chain of reasoning.
        \begin{align*}
          p \rightarrow (q \rightarrow p)
            &\equiv p \rightarrow (\neg q \lor p)
              &\quad
              &\text{by \emph{conditional disintegration}} 
              \\
            &\equiv \neg p \lor (\neg q \lor p)
              &\quad
              &\text{by \emph{conditional disintegration}} 
              \\
            &\equiv \neg p \lor (p \lor \neg q)
              &\quad
              &\text{by \emph{commutativity}} 
              \\
            &\equiv (\neg p \lor p)\lor \neg q
              &\quad
              &\text{by \emph{associativity}} 
              \\
            &\equiv \top\lor \neg q
              &\quad
              &\text{by \emph{complement}} 
              \\
            &\equiv \top
              &\quad
              &\text{by \emph{domination}} 
              \\
        \end{align*}
        
        Therefore, $p \rightarrow (q \rightarrow p)$ is a tautology.
        \end{proof}

\hrulefill

%3d%
      \item
        Show $(p \rightarrow (q \rightarrow r)) \rightarrow ((p \rightarrow q) \rightarrow (p \rightarrow r))$ is a tautology.

        \vspace{2ex}
        \begin{proof}
        Let p, q, and r be propositions. Observe the following chain of reasoning.
        
        \vspace {0.1in}
        
        $(p \rightarrow (q \rightarrow r)) \rightarrow ((p \rightarrow q) \rightarrow (p \rightarrow r))$
        \begin{align*}
          &\equiv (p \rightarrow (\neg q \lor r)) \rightarrow ((\neg p \lor q) \rightarrow (\neg p \lor r))
            &\quad
            &\text{by \emph{conditional disintegration}} \times3 
            \\
          &\equiv (\neg p \lor (\neg q \lor r)) \rightarrow (\neg(\neg p \lor q) \lor (\neg p \lor r))
            &\quad
            &\text{by \emph{conditional disintegration}} \times 2
            \\
          &\equiv (\neg p \lor (\neg q \lor r)) \rightarrow ((\neg(\neg p) \land \neg q) \lor (\neg p \lor r))
            &\quad
            &\text{by \emph{de morgans law}}
            \\
          &\equiv (\neg p \lor (\neg q \lor r)) \rightarrow ((p \land \neg q) \lor (\neg p \lor r))
            &\quad
            &\text{by \emph{double negation}}
            \\
          &\equiv (\neg p \lor (\neg q \lor r)) \rightarrow ((p \lor (\neg p \lor r)) \land (\neg q \lor (\neg p \lor r)))
            &\quad
            &\text{by \emph{distributivity}}
            \\
          &\equiv (\neg p \lor (\neg q \lor r)) \rightarrow (((p \lor \neg p) \lor r) \land (\neg q \lor (\neg p \lor r)))
            &\quad
            &\text{by \emph{associativity}}
            \\
          &\equiv (\neg p \lor (\neg q \lor r)) \rightarrow (((\neg p \lor p) \lor r) \land (\neg q \lor (\neg p \lor r)))
            &\quad
            &\text{by \emph{commutativity}}
            \\
          &\equiv (\neg p \lor (\neg q \lor r)) \rightarrow ((\top \lor r) \land (\neg q \lor (\neg p \lor r)))
            &\quad
            &\text{by \emph{complement}}
            \\
          &\equiv (\neg p \lor (\neg q \lor r)) \rightarrow (\top \land (\neg q \lor (\neg p \lor r)))
            &\quad
            &\text{by \emph{domination}}
            \\
          &\equiv (\neg p \lor (\neg q \lor r)) \rightarrow (\neg q \lor (\neg p \lor r))
            &\quad
            &\text{by \emph{identity}}
            \\
          &\equiv (\neg p \lor (\neg q \lor r)) \rightarrow ((\neg q \lor \neg p) \lor r)
            &\quad
            &\text{by \emph{associativity}}
            \\
          &\equiv (\neg p \lor (\neg q \lor r)) \rightarrow (( \neg p \lor \neg q) \lor r)
            &\quad
            &\text{by \emph{commutativity}}
            \\
          &\equiv (\neg p \lor (\neg q \lor r)) \rightarrow ( \neg p \lor (\neg q \lor r))
            &\quad
            &\text{by \emph{associativity}}
            \\
          &\equiv \neg(\neg p \lor (\neg q \lor r)) \lor ( \neg p \lor (\neg q \lor r))
            &\quad
            &\text{by \emph{conditional disintegration}}
            \\
          &\equiv \top
            &\quad
            &\text{by \emph{complement}}
            \\
        \end{align*}

        Therefore, $(p \rightarrow (q \rightarrow r)) \rightarrow ((p \rightarrow q) \rightarrow (p \rightarrow r))$ is a tautology.
        \end{proof}
    \end{enumerate}  
\pagebreak
%q4%
    \item Show that $\neg$ and $\land$ are sufficient to express \emph{any} proposition.
\begin{proof}
    Observe the following chain of reasoning starting with the formal definition of a proposition.
    
    We say that $r$ is a proposition if $r$ satisfies the following recurrence.
    \begin{enumerate}
        \item[1.] $r = \top$ or $r = \bot$.
        \item[2.] $r = \neg p$, where $p$ is a proposition.
        \item[3.] $r = (p) \land (q)$ where $p$ and $q$ are propositions.
        \item[4.] $r = (p) \lor (q)$ where $p$ and $q$ are propositions.
        \item[5.] $r = (p) \rightarrow (q)$ where $p$ and $q$ are propositions.
        \item[6.] $r = (p) \leftrightarrow (q)$ where $p$ and $q$ are propositions.
    \end{enumerate}
    From this definition, five logical connectives ($\neg$, $\land$, $\lor$,  $\rightarrow$, and $\leftrightarrow$) are sufficient to express any proposition. Thus if we can express $\lor$, $\rightarrow$, and $\leftrightarrow$ with $\neg$ and $\land$ then $\neg$ and $\land$ are sufficient to express any proposition

    Let p and q  represent arbitrary propositions. Observe the following chain of reasoning.
    \begin{enumerate}
        \item $\top$ or $\bot$
            
            This is the base case of the recursive definition and will always be a proposition.
        \item $\neg$
            
            This only uses $\neg$ and thus does not need to be altered to for this question.
        \item $\land$
            
            This only uses $\land$ and thus does not need to be altered to for this question.
        \item $\lor$
            \begin{align*}
                \neg(\neg p \land \neg q)
                    &\equiv \neg(\neg p) \lor \neg(\neg q)
                        &\quad
                        &\text{by \emph{de morgans laws}} 
                        \\
                    &\equiv p \lor q
                        &\quad
                        &\text{by \emph{double negation}}\times2 
                        \\
            \end{align*}
        \item $\rightarrow$
            \begin{align*}
                \neg(p \land \neg q)
                    &\equiv \neg p \lor \neg(\neg q)
                        &\quad
                        &\text{by \emph{de morgans laws}} 
                        \\
                    &\equiv \neg p \lor q
                        &\quad
                        &\text{by \emph{double negation}} 
                        \\
                    &\equiv p \rightarrow q
                        &\quad
                        &\text{by \emph{conditional disintegration}} 
                        \\
            \end{align*}
        \item $\leftrightarrow$
            \begin{align*}
                \neg(q \land \neg p) \land \neg(p \land \neg q)
                    &\equiv (\neg q \lor \neg(\neg p)) \land (\neg p \lor \neg(\neg q))
                        &\quad
                        &\text{by \emph{de morgans laws}} \times 2
                        \\
                    &\equiv (\neg q \lor p)\land (\neg p \lor q)
                        &\quad
                        &\text{by \emph{double negation}} \times2
                        \\
                    &\equiv (q \rightarrow p) \land (p \rightarrow q)
                        &\quad
                        &\text{by \emph{conditional disintegration}} \times2
                        \\
                    &\equiv p \leftrightarrow q
                        &\quad
                        &\text{by \emph{biconditional disintegration}} 
                        \\
            \end{align*}
    \end{enumerate}
    Using these 6 premises, we can additionally break down all sub-propositions to only contain $\neg$ and $\land$ as long as they are of finite length. If they are not of finite length, then they are not propositions.
    
    Therefore, as $\neg$ and $\land$ can express the three other logical connectives, $\neg$ and $\land$ are sufficient to express \emph{any} proposition with this new definition.

    We say that $r$ is a proposition if $r$ satisfies the following recurrence.
    
    \begin{enumerate}
        \item[1.] $r = \top$ or $r = \bot$.
        \item[2.] $r = \neg p$, where $p$ is a proposition.
        \item[3.] $r = (p) \land (q)$ where $p$ and $q$ are propositions.
        \item[4.] $r = (p) \lor (q)$ which can be rewritten as $\neg(\neg p \land \neg q)$ where $p$ and $q$ are propositions.
        \item[5.] $r = (p) \rightarrow (q)$ which can be rewritten as $\neg(p \land \neg q)$ where $p$ and $q$ are propositions.
        \item[6.] $r = (p) \leftrightarrow (q)$ which can be rewritten as $\neg(q \land \neg p) \land \neg(p \land \neg q)$ where $p$ and $q$ are propositions.
    \end{enumerate}
\end{proof}
\pagebreak
\restoregeometry    
    \item Is there a \emph{single connective} capable of expressing \emph{any} proposition?
    
    Yes, there is a single connective capable of expressing any proposition. Looking at this question, all we need is an expression that is able to express $\neg$ and $\land$ because of the logic shown in question 4. This new logical connective I will call ``negand'' with the symbol being $\negland$ and it will act like this: $p \negland q \equiv \neg p \land q$. Now to prove that this single connective can express any proposition, it has to be able to do the function of $\neg$ and the function of $\land$.
\begin{proof}
    Let p and q  represent arbitrary propositions. Observe the following chain of reasoning.
    \begin{enumerate}
        \item $\neg$
            \begin{align*}
                p \negland \top
                    &\equiv \neg p \land \top
                        &\quad
                        &\text{by \emph{definition of negand}} 
                        \\
                    &\equiv \neg p
                        &\quad
                        &\text{by \emph{identity}}
                        \\
            \end{align*}
        \item $\land$
            \begin{align*}
                (p \negland q) \negland q
                    &\equiv \neg(\neg p \land q) \land q
                        &\quad
                        &\text{by \emph{definition of negand}}\times2
                        \\
                    &\equiv (\neg(\neg p) \lor \neg q) \land q
                        &\quad
                        &\text{by \emph{de morgans laws}} 
                        \\
                    &\equiv (p \lor \neg q) \land q
                        &\quad
                        &\text{by \emph{double negation}} 
                        \\
                    &\equiv (p \land q)\lor (\neg q \land q)
                        &\quad
                        &\text{by \emph{definition of negation}}
                        \\
                    &\equiv (p \land q)\lor \bot
                        &\quad
                        &\text{by \emph{complement}}
                        \\
                    &\equiv p \land q
                        &\quad
                        &\text{by \emph{identity}}
                        \\
            \end{align*}

    \end{enumerate}
    Therefore, because every proposition can be expressed by $\land$ and $\neg$, and $\negland$ can form these two logical connectives, every proposition can be expressed by a single connective.
\end{proof}

\end{enumerate}
\end{document}