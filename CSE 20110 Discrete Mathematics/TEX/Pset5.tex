\documentclass{article}
\usepackage{amsmath, amsfonts, amssymb, amstext, amscd, amsthm, bm, faktor, mathrsfs, mathtools, mdframed, thmtools, xfrac, graphicx, multicol}
\usepackage{textcomp}
\usepackage[left=1in, right=1.5in]{geometry}
\usepackage{xcolor}
\renewcommand\qedsymbol{Q.E.D.}
\newcommand{\Emptyset}{\varnothing}
\newcommand{\notsubseteq}{\mathrel{\not\subseteq}}
\newcommand{\union}{\cup\:}
\newcommand{\intersect}{\cap\:}
\newcommand{\defeq}{\coloneqq}
\newcommand{\naturals}{\mathbb{N}}
\newcommand{\s}{\mathbb{S}}
\setlength{\columnseprule}{0.4pt}
\newenvironment{case}[1][Case]
    {\par\textit{#1:}\hfill\break}
    {}

\title{Problem Set 5}
\author{Aaron Wang}
\date{February 27 2024}

\begin{document}

\maketitle

\begin{enumerate}
%q1%
    \item Find and explain the flaw(s) in this argument.
    \begin{mdframed}
      We prove every nonempty set of people all have the same age.
      \begin{proof}
        We denote the age of a person $p$ by $\alpha(p)$.
        \begin{case}[Basis Step]
          Suppose $P = \{p\}$ is a set with one person in it.
          Clearly, all the people in $P$ have the same age as each other.
        \end{case}
        
        \begin{case}[Inductive Step]
          Let $k \in \naturals_+$ and suppose any set of $k$-many people all have the same age.
          Let $P = \{p_1, p_2, \dots p_k, p_{k + 1}\}$ be a set with $k + 1$ people in it.
          Consider $L \defeq \{p_1, \dots p_k\}$ and $R \defeq \{p_2, \dots p_{k + 1}\}$.
          Since $L$ and $R$ both have $k$ people, we know everyone in these sets has the same age by the \emph{inductive hypothesis.}
          % Everyone in $L \intersect R = \set{p_2, \dots p_k}$ has the same age for the same reason.

          Let $\ell, r \in P$.
          If $\ell \in L \land r \in L$, then $\alpha(\ell) = \alpha(r)$.
          Similarly, if $\ell \in R \land r \in R$, then $\alpha(\ell) = \alpha(r)$.
          Now, suppose $\ell \in L \land r \in R$.
          % Then, $\alpha(\ell) = \alpha(p_1)$ and $\alpha(p_{k + 1}) = \alpha(r)$.
          \begin{equation*}
            \alpha(\ell) = \alpha(p_1) = \alpha(p_2) = \alpha(p_{k + 1}) = \alpha(r)
          \end{equation*}
          So, all people in $P$ have the same age.
        \end{case}

        Therefore, everyone on Earth has the same age.
      \end{proof}
    \end{mdframed}
    The argument wants to prove $\forall n (\varphi(n))$ where n is a nonempty set. However, the inductive step iterates over the size of each set rather than each set. In other words, the inductive step considers a set of k people for all k, but not all sets of k people.

    In addition to that, the inductive step falsely assumes that $p_2 \in L$ for every case ($p_2 \notin L$ when k=1). As such, it makes a false claim that $\alpha(p_1) = \alpha(p_2)$.
%q2%
    \item Show that $\forall x (x \neq x \union \{x\})$.
    \begin{proof}
        Let $x$ be an arbitrary set. Let $y \defeq x \union \{x\}$. By definition of union, we know that $y = \{z|z \in x \lor z \in \{x\}\}$. To show that $x\neq y$, we must find an element that is in one set but not the other (extensionality). Looking at $y$, we see that $x\in y$ but looking at x, we see that $x \notin x$. Therefore, since there is an element in y that is not in x, $y \neq x$ and as such $x \neq x \union x$.\\
    \end{proof}
\pagebreak
%q3%
  \item We will work up to a proof of the commutativity of addition on $\naturals$.
    \begin{enumerate}
      \item
        Show $(\forall x \in \naturals)(x + 0 = 0 + x)$.
        \begin{proof} Proof by mathematical induction.
        \begin{case}[Basis Step] 
        Need to show $0+0=0+0$
        \begin{equation*}
            0+0=0+0
        \end{equation*}
        \end{case}
        \begin{case}[Inductive Step]
        Let $x \in \naturals$ and assume that $x+0=0+x$.\\
        Need to show $\s(x)+0=0+\s(x)$
        \begin{align*}
          \s(x)+0
            &=\s(x)
            &\quad
            &\text{By $+$ Rule 1}
              \\
            &=\s(x+0)
            &\quad
            &\text{By $+$ Rule 1}
              \\
            &=\s(0+x)
            &\quad
            &\text{By IH}
              \\
            &=0+\s(x)
            &\quad
            &\text{By $+$ Rule 2}
        \end{align*}
        \end{case}
        Thus by mathematical induction, we can conclude $(\forall x \in \naturals)(x + 0 = 0 + x)$.\\  
        \end{proof}
      \item
        Show $(\forall x, y \in \naturals)(x + \s(y) = \s(y) + x)$.
        \begin{proof}
        Let $x,y \in \naturals$ and assume that $x+y=y+x$.\\
        Need to show $x + \s(y) = \s(y) + x$
        \begin{align*}
          x+\s(y)
            &=\s(x+y)
            &\quad
            &\text{By $+$ Rule 2}
              \\
            &=\s(y+x)
            &\quad
            &\text{By IH}
              \\
            &=y+\s(x)
            &\quad
            &\text{By $+$ Rule 2}
              \\
            &=y+1+x
            &\quad
            &\text{By Theorem 2}
            \\
            &=\s(y)+x
            &\quad
            &\text{By Theorem 1}
        \end{align*} 
        \end{proof}
      \item
        Show $(\forall x, y \in \naturals)(x + y = y + x)$.
        \begin{proof} Proof by mathematical induction.\\\\
        \emph{Basis Step:} We can conclude $x+0=0+x$ by 3a\\
        \emph{Inductive Step:} We can conclude $x + \s(y) = \s(y) + x$ by 3b\\\\
        Thus by mathematical induction, we can conclude $(\forall x, y \in \naturals)(x + y = y + x)$.\\  
        \end{proof}
    \end{enumerate}
%q4%
\pagebreak
  \item Show $(\forall x, y, z \in \naturals)(x \cdot (y + z) = (x \cdot y) + (x \cdot z))$.
    \begin{proof}
    Proof by mathematical induction
    \begin{case}[Basis Step]
    Let $x,y \in \naturals$.
    Need to show $x \cdot (y + 0) = (x \cdot y) + (x \cdot 0)$.
    \begin{align*}
      x \cdot (y + 0)
        &=x \cdot y
        &\quad
        &\text{By $+$ Rule 1}
          \\
        &=(x \cdot y) + 0
        &\quad
        &\text{By $+$ Rule 1}
          \\
        &=(x \cdot y) + (x \cdot 0)
        &\quad
        &\text{By $\cdot$ Rule 1} 
    \end{align*}
    \end{case}
    \begin{case}[Inductive Step]
    Let $x,y,z \in \naturals$ and assume $x \cdot (y + z) = (x \cdot y) + (x \cdot z)$.\\
    Need to show $x \cdot (y + \s(z)) = (x \cdot y) + (x \cdot \s(z))$.
    \begin{align*}
      x \cdot (y + \s(z))
        &=x \cdot \s(y + z)
        &\quad
        &\text{By $+$ Rule 2}
          \\
        &=x \cdot (y + z) + x
        &\quad
        &\text{By $\cdot$ Rule 2}
          \\
        &=(x \cdot y) + (x \cdot z)+x
        &\quad
        &\text{By IH} 
          \\
        &=(x \cdot y) + (x \cdot \s(z))
        &\quad
        &\text{By $\cdot$ Rule 2} 
    \end{align*}
    \end{case}
    Therefore by mathematical induction, we can conclude $(\forall x, y, z \in \naturals)(x \cdot (y + z) = (x \cdot y) + (x \cdot z))$.\\
    \end{proof}
\newgeometry{left=0.5in, right=1in}
\newpage
%q5%
  \item Prove the following statement for all $n \in \naturals$.
    \begin{equation*}
      1+\sum_{i = 0}^{n} 2^i = 2^{n + 1}
    \end{equation*}
    
    \begin{proof}
    Proof by mathematical induction
    \begin{multicols}{2}
    \begin{case}[Basis Step]
    Need to show $1+\sum_{i = 0}^{0} 2^i = 2^{0 + 1}$.
    \begin{align*}
      1+\sum_{i = 0}^{0} 2^i
        &=1+2^0
        &\quad
        &\text{By $\sum$ Rule 1}
          \\
        &=1+1
        &\quad
        &\text{By $n^m$ Rule 1}
          \\
        &=\s(1)
        &\quad
        &\text{By Theorem}
          \\
        &=2
        &\quad
        &\text{By Def of $\s(n)$}
          \\
        &=2+0
        &\quad
        &\text{By $+$ Rule 1}
          \\
        &=0+2
        &\quad
        &\text{By Comm.}
          \\
        &=(2\cdot0)+2
        &\quad
        &\text{By $\cdot$ Rule 1}
          \\
        &=2\cdot\s(0)
        &\quad
        &\text{By $\cdot$ Rule 2}
          \\
        &=2\cdot1
        &\quad
        &\text{By Def of $\s(n)$}
          \\
        &=2\cdot2^0
        &\quad
        &\text{By $n^m$ Rule 1}
          \\
        &=2^{\s(0)}
        &\quad
        &\text{By $n^m$ Rule 2}
          \\
        &=2^{1}
        &\quad
        &\text{By Def of $\s(n)$}
          \\
        &=2^{0+1}
        &\quad
        &\text{By Theorem}
    \end{align*}
    \end{case}
    \columnbreak
    \begin{case}[Inductive Step]
    Assume $1+\sum_{i = 0}^{n} 2^i = 2^{n + 1}$.\\
    Need to show $1+\sum_{i = 0}^{\s(n)} 2^i = 2^{\s(n) + 1}$.
    \begin{align*}
      1+\sum_{i = 0}^{\s(n)} 2^i
        &=1+\sum_{i = 0}^{n} 2^i+2^{\s(n)}
        &\quad
        &\text{By $\sum$ Rule 2}
          \\
        &=2^{n+1}+2^{\s(n)}
        &\quad
        &\text{By IH}
          \\
        &=2^{\s(n)}+2^{\s(n)}
        &\quad
        &\text{By Def of }\s(n) 
          \\
        &=2^{\s(n)}+0+2^{\s(n)}
        &\quad
        &\text{By $+$ Rule 1}
          \\
        &=2^{\s(n)}+(2^{\s(n)}\cdot0)+2^{\s(n)}
        &\quad
        &\text{By $\cdot$ Rule 1}
          \\
        &=2^{\s(n)}+(2^{\s(n)}\cdot\s(0))
        &\quad
        &\text{By $\cdot$ Rule 2}
          \\
        &=2^{\s(n)}+(2^{\s(n)}\cdot1)
        &\quad
        &\text{By Def of }\s(n)
          \\
        &=(2^{\s(n)}\cdot1)+2^{\s(n)}
        &\quad
        &\text{By Comm.}
          \\
        &=(2^{\s(n)}\cdot\s(1))
        &\quad
        &\text{By $\cdot$ Rule 2}
          \\
        &=2^{\s(n)}\cdot2
        &\quad
        &\text{By Def of }\s(n)
          \\
        &=2\cdot2^{\s(n)}
        &\quad
        &\text{By Comm.}
          \\
        &=2^{\s(\s(n))}
        &\quad
        &\text{By $n^m$ Rule 2}
          \\
        &=2^{\s(n)+1}
        &\quad
        &\text{By Def of }\s(n)
    \end{align*}
    \end{case}
    \end{multicols}
    Therefore by mathematical induction, we can conclude $1+\sum_{i = 0}^{n} 2^i = 2^{n + 1}$.\\
    \end{proof}
\restoregeometry  
  \item We say $x$ is $\in$-transitive by definition when $(\forall y \in x)(\forall z \in y)(z \in x)$. 
    Show that every natural number is $\in$-transitive.
    \begin{proof}
    Proof by mathematical induction.
    \begin{case}[Basis Step]
        Let $y,z \in \naturals$\\
        Need to show: 0 is $\in$-transitive or $(\forall y \in 0)(\forall z \in y)(z \in 0)$.\\

        Assume that $y \in 0$ and $z \in y$. By the recursive definition of natural numbers, $0\defeq \Emptyset$. This means that $y \in \Emptyset$. However, as the empty set is empty, $y \notin \Emptyset$. Thus, we can conclude by the explosion theorem that $z \in \Emptyset$. Consequently, 0 is $\in$-transitive.
    \end{case}
    \begin{case}[Inductive Step]
        Let $n,y,z \in \naturals$ and assume $(\forall y \in n)(\forall z \in y)(z \in n)$.\\
        Need to show $(\forall y \in \s(n))(\forall z \in y)(z \in \s(n))$.\\
        Let $y \in \s(n)$ and $z \in y$. First, observe that by definition, $\s(n)=n\union\{n\}$. This means that $y \in n \lor y \in \{n\}$. Now let's consider these as two cases.\\
        \begin{case}[$y\in n$]
             We can see that $z \in n$ by inductive hypothesis. 
        \end{case}
        \begin{case}[$y \in \{n\}$]
             Since $y \in \{n\}$, we can see that $y=n$. Since we know that $z \in y$, we know that $z \in n$ by extensionality. Furthermore, since $\s(n)=n\union \{n\}$, we know that $z \in \s(n)$.\\\\
        \end{case}
    \end{case}
    As such, by mathematical induction, we can conclude that $(\forall y \in x)(\forall z \in y)(z \in x)$.\\
    \end{proof}
\end{enumerate}

\end{document}
