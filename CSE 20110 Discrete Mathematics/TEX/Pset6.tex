\documentclass{article}
\usepackage{amsmath, amsfonts, amssymb, amstext, amscd, amsthm, bm, faktor, mathrsfs, mathtools, mdframed, thmtools, xfrac, graphicx, multicol, stmaryrd}
\usepackage{textcomp}
\usepackage[left=1in, right=1.5in]{geometry}
\usepackage{xcolor}

\renewcommand\qedsymbol{Q.E.D.}
\newcommand{\Emptyset}{\varnothing}
\newcommand{\notsubseteq}{\mathrel{\not\subseteq}}
\newcommand{\union}{\cup\:}
\newcommand{\intersect}{\cap\:}
\newcommand{\defeq}{\coloneqq}
\newcommand{\naturals}{\mathbb{N}}
\newcommand{\s}{\mathbb{S}}
\newcommand{\divides}{\:\mathbb{|}\: }
\newcommand{\notdivides}{\:\mathbb{|}\!\!\backslash\: }
\newcommand{\integers}{\mathbb{Z}}
\newcommand{\fibonacci}{\mathcal{F}}
\newenvironment{case}[1][Case]
    {\par\textit{#1:}\hfill\break}
    {}

\title{Problem Set 6}
\author{Aaron Wang}
\date{March 5 2024}

\begin{document}

\maketitle

\begin{enumerate}
%q1%
    \item
    \begin{enumerate}
%1a%
      \item Show that $(c \neq 0 \land ac \divides bc) \implies (a \divides b)$ for all $a, b, c \in \integers$.
        \begin{proof}
        Let $a,b,c\in \integers$ and assume $c \neq 0 \land ac \divides bc$. From $ac \divides bc$, we know there is a $k$ such that $(k \in \integers)$ $(ack=bc)$ by definition of divides. Using commutativity, we get $akc=bc$. Since we know $c \neq 0 \land akc=bc$, by multiplicative cancellation, we can conclude $ak=b$ which implies $a \divides b$.\\
        \end{proof}
%1b%
      \item Show that $(n \divides x \land n \divides y) \implies (n \divides ax + by)$ for all $n, x, y, a, b \in \integers$.
        \begin{proof}
        Let $a,b,n,x,y \in \integers$ and assume $n \divides x \land n \divides y$. By definition of division, we know there is a $g$ such that $(g \in \integers)$ $(ng=x)$ and some value $h$ for which $nh=y$. Now let $k=ag+bh$. Thus, we know that $n \cdot k = n \cdot (ag+bh)$. By the definition of division (and distribution and commutativity), we know that $n|ang+bnh$. Finally, by substitution, we know that $n|ax+by$.\\
        \end{proof}
    \end{enumerate}
%q2%
    \item For all $z \in \integers$, show that $z$ is even implies $z$ is not odd.
    \begin{enumerate}
        \item []
        \begin{proof}
        Let $z \in \integers$ and assume that $z$ is even. By definition of even $2 \divides z$ and by definition of divides there is a $k$ such that $(k \in \integers)(2k=z)$. Towards a contradiction, assume that $z$ is odd. By definition of odd, $2 \divides z-1$ and as such, there is a $h$ such that $(h \in \integers)(2h=z-1)$. Using substitution, we get $2h=2k-1$. Subtract $2k$ from and multiply by $-1$ for both sides and we get $2k-2h=1$. Furthermore, by distributivity, we get $2(k-h)=1$. Since $(k-h) \in \integers$, we conclude that $2 \mid 1$ which by the Absolute Monotonicity of Divisibility implies $|2|\leq|1|$ $\lightning$. Consequently, we conclude that $z$ is even implies $z$ is not odd.\\
    \end{proof}
    \end{enumerate}
    
\pagebreak
%q3%
  \item
    \begin{enumerate}
      \item For all $n \in \naturals$, show that $n$ is even implies $n + 1$ is odd.
        \begin{proof}
          Let $n \in \naturals$ and assume that $n$ is even. This means that $2 \mid n$. Since $n=(n+1)-1$ we can conclude that $2 \mid (n+1)-1$ and consequently we know that $n + 1$ is odd.\\
        \end{proof}
      \item For all $n \in \naturals$, show that $n$ is odd implies $n + 1$ is even.
        \begin{proof}
          Let $n \in \naturals$ and assume that $n$ is odd. This means that $2 \mid n-1$. Note that $2\mid2$ $(2\cdot k=2 \text{ when } k = 1)$. Thus, $2 \mid n-1+2$ by (1b). Also, observe that $n-1+2=n+1$. Consequently, $2\mid n+1$. Therefore, $n + 1$ is even.\\
        \end{proof}
    \end{enumerate}
%q4%
  \item
    Show that $3 \divides n^3 - n$ for all $n \in \naturals$.
    \begin{proof} Proof by mathematical induction
        \begin{case}[Basis Step]
            Need to show $3\mid0^3-0$.\\
            Since $(0^3-0=0) \land (3\mid0)$ we can conclude $3\mid0^3-0$.
        \end{case}
        \begin{case}[Inductive Step]
            Let $k \in \naturals$ and assume that $3 \divides k^3 - k$.\\
            Need to show $3\mid\s(k)^3-\s(k)$.\\
            $\s(k)^3-\s(k)=(k+1)^3-(k+1)=k^3+3k^2+3k+1-k-1=k^3-k+3k^2+3k=(k^3-k)+3(k^2+k)$\\
            Let $b=k^2+k$ and let $a=1$. Now $(k^3-k)+3(k^2+k)=a(k^3-k)+3b$.\\
            Since $(3 \mid k^3-k) \land (3\mid3)\land a,b\in\integers$ we can conclude $3 \mid a(k^3-k)+3b$ by (Problem 1b).\\
            Since $(\s(k)^3-\s(k)=a(k^3-k)+3b) \land (3 \mid a(k^3-k)+3b)$, we can conclude $3\mid\s(k)^3-\s(k)$.\\
            Therefore $3\mid\s(k)^3-\s(k)$.\\\\
        \end{case}
        Thus by mathematical induction, we can conclude $3 \divides n^3 - n$ for all $n \in \naturals$. 
    \end{proof}
\pagebreak
%q5%
  \item The Fibonacci sequence is the recursive function $\fibonacci: \naturals \to \naturals$ below.
    \begin{align*}
        \fibonacci(0) &\defeq 0 \\
        \fibonacci(1) &\defeq 1 \\
        \fibonacci(n + 2) &\defeq \fibonacci(n + 1) + \fibonacci(n)
    \end{align*}
    Show that $1 + \displaystyle\sum_{i = 0}^{n}\fibonacci(i) = \fibonacci(n + 2)$ for all $n \in \naturals$.
    \begin{proof} Proof by mathematical induction.
        \begin{case}[Basis Step] 
        Need to show $1 + \displaystyle\sum_{i = 0}^{0}\fibonacci(i) = \fibonacci(0+2)$
        \begin{align*}
          1 + \displaystyle\sum_{i = 0}^{0}\fibonacci(i)
            &=1+\fibonacci(0)
            &\quad
            &\text{By $\sum$ Rule 1}
              \\
            &=\fibonacci(1)+\fibonacci(0)
            &\quad
            &\text{By definition of } \fibonacci(1)
              \\
            &=\fibonacci(2)
            &\quad
            &\text{By definition of } \fibonacci(n+2)
              \\
            &=\fibonacci(0+2)
            &\quad
            &\text{By addition}
              \\
        \end{align*}
        \end{case}
        \begin{case}[Inductive Step]
        Let $k \in \naturals$ and assume that $1 + \displaystyle\sum_{i = 0}^{k}\fibonacci(i) = \fibonacci(k + 2)$.\\
        Need to show $1 + \displaystyle\sum_{i = 0}^{\s(k)}\fibonacci(i) = \fibonacci(\s(k) + 2)$.
        \begin{align*}
          1 + \displaystyle\sum_{i = 0}^{\s(k)}\fibonacci(i)
            &=1+\displaystyle\sum_{i = 0}^{k}\fibonacci(i)+\fibonacci(\s(k))
            &\quad
            &\text{By $\sum$ Rule 1}
              \\
            &=\fibonacci(k + 2)+\fibonacci(\s(k))
            &\quad
            &\text{By IH}
              \\
            &=\fibonacci(k + \s(1))+\fibonacci(\s(k))
            &\quad
            &\text{By definition of }\s(n)
              \\
            &=\fibonacci(k + 1+1)+\fibonacci(\s(k))
            &\quad
            &\text{By Theorem}
              \\
            &=\fibonacci(\s(k) + 1)+\fibonacci(\s(k))
            &\quad
            &\text{By definition of }\s(n)
              \\
            &=\fibonacci(\s(k) + 2)
            &\quad
            &\text{By definition of } \fibonacci(n+2)
              \\
        \end{align*}
        Therefore we have concluded $1 + \displaystyle\sum_{i = 0}^{\s(k)}\fibonacci(i) = \fibonacci(\s(k) + 2)$.\\
        \end{case}
        Thus by mathematical induction, we can conclude $1 + \displaystyle\sum_{i = 0}^{n}\fibonacci(i) = \fibonacci(n + 2)$ for all $n \in \naturals$.\\  
        \end{proof}
\end{enumerate}

\end{document}
