\documentclass{article}
\usepackage{amsmath, amsfonts, amssymb, amstext, mathtools}
\usepackage{textcomp}
\usepackage[left=1in, right=1.5in]{geometry}
\usepackage{xcolor}

\newcommand{\Emptyset}{\varnothing}
\newcommand{\notsubseteq}{\mathrel{\not\subseteq}}
\newcommand{\union}{\cup}
\newcommand{\intersect}{\cap}
\newcommand{\defeq}{\coloneqq}
\newcommand{\naturals}{\mathbb{N}}
\newcommand{\s}{\mathbb{S}}
\newcommand{\power}{\mathbb{P}}
\newcommand{\expect}{E}
\newcommand{\var}{Var}

\title{Homework 10}
\author{Aaron Wang}
\date{December 6 2024}

\begin{document}
\maketitle
\begin{enumerate}

    \item $Y_1$ and $Y_2$ denoted the proportions of time during which employees $I$ and $II$ actually performed their assigned tasks during a workday. The joint density of $Y_1$ and $Y_2$ is given by
    \[
    f(y_1,y_2) = 
    \begin{cases}
    y_1+y_2 & 0 \leq y_1 \leq 1 \text{ and } 0 \leq y_2 \leq 1 \\
    0 & \text{otherwise}
    \end{cases}
    \]
    \begin{enumerate}
        \item Find the marginal density functions for $Y_1$ and $Y_2$.
        \[
            f_{Y_1}(y_1) = \int_{-\infty}^{\infty} f(y_1,y_2) dy_2
        \]
        \[
            \int_{0}^{1} \big(y_1+y_2\big) dy_2 = y_1y_2+\frac{y_2^2}{2}\Big|_0^1 = y_1+\frac{1}{2}
        \]
        \textcolor{red}{\[
            f_{Y_1}(y_1) = y_1+\frac{1}{2} \text{ for } 0 \leq y_1 \leq 1
        \]}
        Follow the same process to get:
        \textcolor{red}{\[
            f_{Y_2}(y_2) = y_2+\frac{1}{2} \text{ for } 0 \leq y_2 \leq 1
        \]}

        \item Find $P(Y_1 \geq 0.5|Y_2 \geq 0.5)$.
        \[
            P(Y_1 \geq 0.5 \intersect Y_2 \geq 0.5) = \int_{0.5}^1 \int_{0.5}^1 \big(y_1+y_2\big) dy_1dy_2 = 0.375
        \]
        \[
            P(Y_2 \geq 0.5) = \int_{0.5}^1 y_2+\frac{1}{2} = 0.625
        \]
        \[
            P(Y_1 \geq 0.5|Y_2 \geq 0.5) = \frac{P(Y_1 \geq 0.5 \intersect Y_2 \geq 0.5)}{P(Y_2 \geq 0.5)} = \frac{0.375}{0.625}=\textcolor{red}{0.6}
        \]
        \item If employee II spends exactly $50\%$ of the dayworking on assigned duties, find the
        probability that employee I spends more than $75\%$ of the day working on similar duties.
        \[
            f_{Y_1|Y_2}(y_1|y_2)=\frac{f(y_1,y_2)}{f_{Y_2}(y_2)}=\frac{y_1+y_2}{0.5+y_2}
        \]
        \[
            P(Y_1 \geq 0.75 | Y_2 = 0.5) = \int_{0.75}^1 f_{Y_1|Y_2}(y_1|0.5) dy_1 = \int_{0.75}^1 \big(y_1+0.5\big) dy_1 = \textcolor{red}{0.34375}
        \]
    \end{enumerate}
\pagebreak
    \item Assume that $Y$ denotes the number of bacteria per cubic centimeter in a particular liquid and that $Y$ has a Poisson distribution with parameter $Z$. Further assume that Z varies from location to location and has a gamma distribution with parameters $\alpha$ and $\beta$, where $\alpha$ is a positive integer. If we randomly select a location, what is the
    \begin{enumerate}
        \item expected number of bacteria per cubic centimeter?
        \[
            E[Y] = E[E[Y|Z]]=E[Z]=\textcolor{red}{\frac{\alpha}{\beta}}
        \]
        \item standard deviation of the number of bacteria per cubic centimeter?
        \[
            Var[Y] = Var[E[Y|Z]]+E[Var[Y|Z]]= Var[Z]+E[Z] = \frac{\alpha}{\beta^2}+\frac{\alpha}{\beta}
        \]
        \[
            STDEV[Y]=\sqrt{Var[Y]}=\textcolor{red}{\sqrt{\frac{\alpha}{\beta^2}+\frac{\alpha}{\beta}}}
        \]
    \end{enumerate}
    \item A random variable $Y$ has the density function $f(y) = e^y$ if $y < 0$, and $f(y) = 0$ elsewhere.
    \begin{enumerate}
        \item Find $E[e^{3Y/2}]$.
        \[
            E[e^{3Y/2}]= \int_{-\infty}^{\infty} e^{3y/2} \cdot f(y)dy = \int_{-\infty}^{0} e^{3y/2} \cdot e^y dy = \int_{-\infty}^{0} e^{5y/2} dy = \textcolor{red}{\frac{2}{5}}
        \]
        \item Find the moment-generating function for $Y$.
        \[
            M(t)=\int_{-\infty}^{\infty}e^{ty}\cdot f(y)dy = \int_{-\infty}^{0}e^{ty}\cdot e^y dy = \int_{-\infty}^{0}e^{(t+1)y}dy = \frac{e^{(t+1)y}dy}{t+1}\Bigg|_{-\infty}^{0}=\textcolor{red}{\frac{1}{t+1}}
        \]
        \item Use the moment-generating function to find $Var(Y)$.
        \[
            E[Y] = \frac{d}{dt}M(t)\Big|_{t=0}=\frac{d}{dt}{\frac{1}{t+1}}\Big|_{t=0}={\frac{-1}{(t+1)^2}}\Big|_{t=0} = -1
        \]
        \[
            E[Y^2] = \frac{d^2}{dt^2}M(t)\Big|_{t=0}=\frac{d^2}{dt^2}{\frac{1}{t+1}}\Big|_{t=0}=\frac{d}{dt}{\frac{-1}{(t+1)^2}}\Big|_{t=0} = {\frac{2}{(t+1)^3}}\Big|_{t=0} = 2
        \]
        \[
            Var[Y] = E[Y^2] - E[Y]^2 = 2- (-1)^2 = \textcolor{red}{1}
        \]
    \end{enumerate}
\pagebreak
    \item Consider the bivariate random variable with density
    \[
    f (y_1, y_2) =
    \begin{cases}
    2 & 0 \leq y_1 \leq 1, 0 \leq y_2 \leq 1, 0 \leq y_1 + y_2 \leq 1\\
    0 & \text{otherwise}
    \end{cases}
    \]
    \begin{enumerate}
        \item Find $E(Y_1 + Y_2)$.
    \[
        f_{Y_1}(y_1) = \int_0^{1-y_1} 2 dy_2 = 2\Big|_0^{1-y_1} = 2-2y_1
    \]
    \[
        E[Y_1] = \int_0^1 \big( y_1(2-2y_1) \big) dy_1 = \int_0^1 \big( 2y_1-2y_1^2 \big) dy_1 = \frac{1}{3}
    \]
    Do the same for $Y_2$ to discover $E[Y_2] = \frac{1}{3}$
    \[
        E[Y_1 + Y_2] = E[Y_1] + E[Y_2] = \frac{1}{3} + \frac{1}{3} = \textcolor{red}{\frac{2}{3}}
    \]
    \item Find $Var(Y_1 + Y_2)$.
    \[
        E[Y_1^2] = \int_0^1 \big( y_1^2(2-2y_1) \big) dy_1 = \int_0^1 \big( 2y_1^2-2y_1^3 \big) dy_1 = \frac{1}{6}
    \]
    \[
        Var[Y_1] = E[Y_1^2] - E[Y_1]^2 = \frac{1}{6}-\frac{1}{9} = \frac{1}{18}
    \]
    Do the same for $Y_2$ to discover $Var[Y_2] = \frac{1}{18}$
    \[
        E[XY] = \int_0^1 \int_0^{1-y_1} 2y_1y_2 dy_2dy_1 = \frac{1}{12}
    \]
    \[
        Cov[Y_1,Y_2]=E[XY] - E[X] \cdot E[Y] = \frac{1}{12} - \frac{1}{3} \cdot \frac{1}{3} = \frac{-1}{36}
    \]
    \[
        Var[Y_1+Y_2] = Var[Y_1] + Var[Y_2] + 2Cov[X,Y] = \frac{1}{18} + \frac{1}{18} +2 \Big(\frac{-1}{36}\Big) = \textcolor{red}{\frac{1}{18}}
    \]
    \end{enumerate}
\pagebreak
    \item You begin with a stick of length 1 and break it at point, chosen uniformly at random. You then take the left piece and break it once again at a uniformly random chosen point. What is the expectation and variance of the length of left piece after the breaking.
    \begin{center}
        Let $Y_1 \sim $\;Unif$[0,1]$ represent the first breaking and $Y_2 \sim $\;Unif$[0,y_1]$ represent the second.
    \end{center}
    \[
        E[Y_2] = E[E[Y_2|Y_1]]=E[\frac{Y_1}{2}]=\frac{1}{2}E[Y_1] = \frac{1}{2}\Big(\frac{1}{2}\Big) = \textcolor{red}{\frac{1}{4}}
    \]
    \[
        E[Y_1^2] = \int_0^1y_1^2dy_1 = \frac{1}{3}
    \]
    \[
        Var[Y_2] = Var[E[Y_1|Y_2]] + E[Var[Y_1|Y_2]] = Var[\frac{Y_1}{2}] + E[\frac{Y_1^2}{12}] 
    \]
    \[
        = \frac{1}{4}Var[Y_1] + \frac{1}{12}E[Y_1^2] = \frac{1}{4}\Big(\frac{1}{12}\Big) + \frac{1}{12}\Big(\frac{1}{3}\Big) = \textcolor{red}{\frac{7}{144}}
    \]
    


\end{enumerate}
\end{document}